%=========Checkpoints=========%
%
% COMPACTNESS => SEQUENTIAL COMPACTNESS

\chapter{Metric Spaces}

\begin{conv}
	Unless stated otherwise, assume the following:
	\begin{assmplist}
		\item $X$, $Y$ will denote metric spaces.
		
		\item Subsets of metric spaces will be seen as metric subspaces.
		
		\item For $x\in X$ and $r > 0$, we'll use
		\begin{assmplist}
			\item $B(x, r) := \{y\in X : d(y, x) < r\}$, and
			\item $D(x, r) := \{y\in X : d(y, x)\le r\}$.
		\end{assmplist}
		Sometimes, we'll also denote these by $B_r(x)$ and $D_r(x)$.
		
		\item A metric space will also be considered a topological space under the induced topology.
		
		\item The diameter of a subset $A$ of a metric space will be denoted by $\delta(A)$.
	\end{assmplist}
\end{conv}

\section{General}
	
	The triangle inequality immediately yields:
	
	\begin{lem}
		Metric is continuous. Further, if $E\subseteq X$, then $x\mapsto d(x, E)$ is also continuous.
	\end{lem}
	
	\begin{rmk}
		Note that $d(x, \emptyset) = +\infty$ for all $x$.
	\end{rmk}
	
	
	\begin{lem}\label{LEM: separable metric spaces are second countable}
		\leavevmode
		\begin{mylist}
			\item Metric spaces are first countable.
			\item Separable metric spaces are second countable.
		\end{mylist}
	\end{lem}
	
	\begin{proof}
		\begin{mylist}
			\item $B_{1/n}(x)$'s forms a local base at $x$.
			
			\item Let $S$ be a countable dense subset of $X$. Then $\bigcup_{x\in S}\{ B_{1/n}(x) : n\ge 1 \}$ forms a countable base:
			\begin{subproof}
				Consider $B_{1/n}(y)$. Let $x\in B_{1/2n}(y)\cap S$. Then $y\in B_{1/2n}(x)\subseteq B_{1/n}(y)$.\qedhere
			\end{subproof}
		\end{mylist}
	\end{proof}

	Let $E\subseteq X$ and $x\in X$. Then a point $y\in E$ is called \defn{a point of best approximation} for $x$ in $Y$ iff $d(x, y) = d(x, E)$.
	
	

\section{Uniform Properties}

Uniform properties encompass things like uniform continuity and Cauchy sequences.



\begin{lem}\label{LEM: closed subsets in complete spaces}
	In a complete space, closed subspaces are precisely the complete ones.
\end{lem}


\begin{lem}
	In a metric space, each sequence has a Cauchy subsequence $\iff$ the space is totally bounded.
\end{lem}

\begin{proof}
	``$\Rightarrow$'': Consider a sequence $(x_i)$. Take an infinite subset $I_1$ of the indices such that $\{x_i : i\in I_1\}$ lies in a ball of diameter $1$. Having chosen $I_n$, choose\myMargin{
		\DC used.}
	an infinite subset $I_{n + 1}\subseteq I_n$ such that $\{x_i : i\in I_{n + 1}\}$ lies in a ball of diameter $1/(n + 1)$. (This is possible since the the space is \ul{totally bounded}.) Now, choose $i_n\in I_n$ such that $(i_n)$ is increasing. Then $(x_{i_n})_n$ forms a Cauchy sequence, for for $n > m\ge N$, we have $d(x_{i_m}, x_{i_n}) < 1/N$.
	
	``$\Leftarrow$'': Suppose $X$ is not totally bounded so that take an $\epsilon > 0$ such that no finitely many balls of radius $\epsilon$ can ever cover $X$. Let $x_1\in X$.\footnote{
		Note that $X$ has got to be enoenmtpy.}
	Having chosen $x_1, \ldots, x_n$, choose\myMargin{
		\DC used.}
	$x_{n + 1}\in X\setminus\bigcup_{i = 1}^n B_\epsilon(x_i)$. Then $(x_i)$ is non-Cauchy sequence, for $d(x_i, x_j)\ge \epsilon$ for all $i\ne j$.
\end{proof}

\begin{rmk}
	The discrete metric on an infinite set, which is not totally bounded, contains sequences with no Cauchy subsequences.
\end{rmk}



\subsection{Completion of metric spaces}

	A \defn{completion} of $X$ is a complete metric space $\hat X$ together with an isometry $\iota\colon X\to \hat X$ such that any isometry $X\to Y$ into a complete metric space $Y$ factors uniquely through $\iota$ via an isometry:
	\[
	\begin{tikzcd}[column sep=small]
		X\ar[rr]\ar[dr, "\iota"'] & & Y\\
		& \hat X\ar[ur, dashed] & 
	\end{tikzcd}
	\]
	
	\begin{cor}
		If $X$ is complete, then $\id\colon X\to X$ is a completion of $X$.
	\end{cor}
	
	\begin{prp}\label{PRP: completion of metric spaces}
		Each metric space admits a completion, which is unique up to bi-isometries.\footnote{
			A bi-isometry is a bijective isometry whose inverse is also an isometry.
		}
	\end{prp}
	
	\noindent The following easy facts will be employed to simplify the proof:
	
	\begin{lem}\label{LEM: some properties of Cauchy sequences}
		\leavevmode
		\begin{mylist}
			\item If $(x_i)$ is Cauchy and $(\alpha_i)\in\mathbb R^+$, then there exists a subsequence $(x_{i_j})$ such that for each $N$, we have $d(x_{i_j}, x_{i_k}) < \alpha_N$ whenever $j, k\ge N$.
			
			\item A Cauchy sequence converges iff any of its subsequence converges.
		\end{mylist}
	\end{lem}
	
	\begin{proof}[Proof of \myRef{PRP: completion of metric spaces}]
		The uniqueness follows by the usual categorical argument. Let's show the existence of a completion of $X$. Define $\hat X$ to be set of the Cauchy sequences in $X$ modded out by the following equivalence relation:
		\[
		(x_i)\sim (y_i) \text{ iff } d(x_i, y_i)\to 0
		\]
		The following defines a well-defined metric on $\hat X$:
		\[
		\hat d\bigl( (x_i), (y_i) \bigr) := \lim\nolimits_i d(x_i, y_i)
		\]
		We show that $\hat X$ is complete:
		\begin{subproof}
			Let $\bigl(\,\overline{x^{(n)}}\,\bigr)$ be Cauchy in $\hat X$, where each $x^{(n)}$ is a Cauchy sequence $\bigl(x^{(n)}_i\bigr)$ in $X$.\myMargin{
				\CC used.
			} Noting that each subsequence of a Cauchy sequence in $X$ is related to the parent sequence, and due to \myRef{LEM: some properties of Cauchy sequences}, we may \wlogg assume:
			\begin{mylist}
				\item $n \ge m\implies \hat d\bigl(\, \overline{x^{(n)}}, \overline{x^{(m)}}\, \bigr) < 1/m$.
				
				\item For each $n$, we have $j\ge i\implies d\bigl(x^{(n)}_j, x^{(n)}_i\bigr) < 1/i$.
			\end{mylist}
			Now, it follows that the diagonal sequence $\bigl(x^{(i)}_i\bigr)$ is Cauchy:
			\begin{alignat*}{2}
				d\bigl(x^{(j)}_j, x^{(i)}_i\bigr)
				& \le d\bigl(x^{(j)}_j, x^{(i)}_j\bigr) + d\bigl(x^{(i)}_j, x^{(i)}_i\bigr)\\
				& < d\bigl(x^{(j)}_j, x^{(j)}_k\bigr) + d\bigl(x^{(j)}_k, x^{(i)}_k\bigr) + d\bigl(x^{(i)}_k, x^{(i)}_j\bigr) & \text{($k$ arbitrary)}\\
				& \phantom{{} < d\bigl(x^{(j)}_j, x^{(j)}_k\bigr) + d\bigl(x^{(j)}_k, x^{(i)}_k\bigr) abc} {}+ 1/i & \text{(letting $j\ge i$)}\\
				& < 1/j + d\bigl(x^{(j)}_k, x^{(i)}_k\bigr) + 1/j +1/i & \text{(letting $k\ge j$)}\\
				& \le 2/i + \hat d\bigl(\, \overline{x^{(j)}},\, \overline{x^{(i)}}\, \bigr) + 1/i & \text{(taking $k\to \infty$)}\\
				& < 2/j + 2/i & \text{(since $j\ge i$)}
 			\end{alignat*}
			Also, $\overline{x^{(n)}}\to \overline{\bigl(x^{(i)}_i\bigr)}$ in $\hat X$:
			\begin{alignat*}{2}
				d\bigl( x^{(n)}_i, x^{(i)}_i \bigr)
				& \le d\bigl( x^{(n)}_i, x^{(n)}_j \bigr) + d\bigl( x^{(n)}_j, x^{(i)}_j \bigr) + d\bigl( x^{(i)}_j, x^{(i)}_i \bigr)\\
				& < 1/i + d\bigl( x^{(n)}_j, x^{(i)}_i \bigr) + 1/i & \text{(letting $j\ge i$)}\\
				& < 2/i + d\bigl(\, \overline{x^{(n)}},\,\overline{x^{(i)}} \,\bigr) & \text{(taking $j\to\infty$)}\\
				& < 3/i
			\end{alignat*}
		\end{subproof}

		\noindent We now check for the universal property:
		\begin{subproof}
			Note that $\iota\colon X\to\hat X$ given by $x\mapsto (x, x, \ldots)$ is an isometry. Let $f\colon X\to Y$ be another isometry with $Y$ being complete. Suppose it does factor through $\iota$ via an isometry $\hat f$:
			\[
			\begin{tikzcd}[column sep=small]
				X\ar[dr, "\iota"']\ar[rr, "f"] && Y\\
				& \hat X\ar[ur, dashed, "\hat f"'] &
			\end{tikzcd}
			\]
			This in turn determines $\hat f$ uniquely:
			\begin{subproof}
				Let $\overline{(x_i)}\in\hat X$, where $(x_i)$ is Cauchy in $X$. Clearly, $\iota(x_i)\to\overline{(x_i)}$ so that
				\[
				f(x_i)\to \hat f\bigl(\, \overline{(x_i)} \,\bigr) \addtag\label{EQN: metric completion proof}
				\]
				as \ul{$\hat f$ is continuous} and \ul{$\hat f\circ \iota = f$}.
			\end{subproof}
			We now show that \myRef{EQN: metric completion proof} indeed defines a factoring of $f$ via $\iota$:
			\begin{prooflist}
				\item $\hat f$ is well-defined:
				\begin{rmklist}
					\item If $(x_i)$ is Cauchy in $X$, then since \ul{$f$ is an isometry} and \ul{$Y$ is complete}, $(f(x_i))$ is convergent in $Y$.
					
					\item If $(x_i)$ and $(y_i)$ are equivalent Cauchy sequences in $X$, then $d(x_i, y_i)\to 0\wimplies d(f(x_i), f(y_i))\to 0$ (since \ul{$f$ an isometry}) so that $d\bigl(\lim_i f(x_i), \lim_i f(y_i)\bigr) = 0$.\footnote{Recall the component-wise convergence in product topology.}
				\end{rmklist}
				
				\item $\hat f$ is an isometry:
				\begin{alignat*}{2}
					d\Bigl( \hat f\bigl(\, \overline{(x_i)} \,\bigr),\, \hat f\bigl(\, \overline{(y_i)} \,\bigr) \Bigr)
					& = d\bigl(\lim\nolimits_i f(x_i), \lim\nolimits_i f(y_i)\bigr)\\
					& = \lim\nolimits_i d(f(x_i), f(y_i))\\
					& = \lim\nolimits_i d(x_i, y_i) & \text{(\ul{$f$ is an isometry})}\\
					& = \hat d\bigl(\, \overline{(x_i)},\, \overline{(y_i)} \,\bigr)
				\end{alignat*}
				
				\item Finally, $\hat f\circ \iota = f$ is clear.\qedhere
			\end{prooflist}
		\end{subproof}
	\end{proof}
	
	\begin{comment}
		\begin{rmk}
		Notice that while proving the uniqueness of $\hat f$, full power of $\hat f$'s isometry was not used, just the continuity. Thus, we can strengthen the universal property like so: Any $f\colon X\to Y$ with $Y$ complete factors uniquely via $\iota$, say through $\hat f$, and this $\hat f$ is further an isometry.
	\end{rmk}
	\end{comment}
	
	
	\begin{prp}
		Any space is dense in its completion.\myMargin{
			Ponder: Can this be taken as an alternative universal property?
		}
	\end{prp}
	
	\begin{proof}
		Let $\iota\colon X\to \hat X$ be a completion of $X$. Now, the restriction $\iota|\colon X\to \clos{\iota(X)}$ is also a completion of $\hat X$:
		\begin{prooflist}
			\item $\clos{\iota(X)}$ is complete due to \myRef{LEM: closed subsets in complete spaces} and $\iota|$ is still an isometry.
			
			\item Let $f\colon X\to Y$ be an isometry with $Y$ complete. Then $f$ factors through $\iota$ which induces a factoring through $\iota|$ as well:
			\[
			\begin{tikzcd}
				& X\ar[dl, "\iota|"']\ar[d, "\iota"]\ar[dr, "f"] &\\
				\clos{\iota(X)}\ar[r, hook] & \hat X\ar[r, dashed] & Y
			\end{tikzcd}
			\]
			For uniqueness, just note that any factoring $\hat f$ of $f$ through $\iota|$ is uniquely determined on $\iota(X)$ which is dense in $\clos{\iota(X)}$, thereby also getting determined on $\clos{\iota(X)}$ (by \myRef{PRP: cont func into Hausdorff uniquely by its vals on dense subset}):
			\[
			\begin{tikzcd}
				& X\ar[dl]\ar[d, "\iota|"]\ar[dr, "f"] &\\
				\iota(X)\ar[r, hook] & \clos{\iota(X)}\ar[r, "\hat f"'] & Y
			\end{tikzcd}
			\]
		\end{prooflist}
		
		Since \ul{$\hat X$ is a completion}, there exists an isometry $\alpha$ such that \tfdc:
		\[
		\begin{tikzcd}[column sep=small]
			& X\ar[dl, "\iota|"']\ar[dr, "\iota"] &\\
			\clos{\iota(X)}\ar[rr, hook, "\operatorname{incl}", shift left] & & \hat X\ar[ll, dashed, "\alpha", shift left]
		\end{tikzcd}
		\]
		It follows that $\iota|$ factors through itself via $\alpha\circ\operatorname{incl}$, so that it is precisely $\id_{\clos{\iota(X)}}$ (since \ul{$\clos{\iota(X)}$ is a completion}) $\wimplies$ $\operatorname{incl}$ is surjective $\wimplies$ $\clos{\iota(X)} = \hat X$.
	\end{proof}
	
	
	From \myRef{LEM: denseness is transitive}, it now immediately follows that:
	
	\begin{cor}
		Completion preserves separability.
	\end{cor}
	
	
\subsection{Metric equivalences}

	\begin{prp}
		For metrics on a given set, we have:
		\begin{center}
			Uniform equivalence $\implies$ $\id$ is uniformly continuous in both directions $\implies$ same Cauchy sequences $\implies$ same convergence $\iff$ topological equivalence.
		\end{center}
	\end{prp}
	
	\begin{proof}
		The first two implications are trivial and the last follows from \myRef{COR: first countable T1 topologgies via their sequences}. For the penultimate, just note that $x_i\to c$ iff $x_1, c, x_2, c, \ldots$ is Cauchy.
	\end{proof}
	
	\begin{rmk}
		None of the converses are true. Let $f\colon X\to X$ be a \homeo which thus induces a topologically equivalent metric on $X$ (see \myRef{LEM: to show counters for equiv of metrics}).
		\begin{mylist}
			\item Let $f$, $f^{-1}$ be uniformly continuous and $f$ not be Lipschitz (for instance, $f\colon x\mapsto \sqrt x$ on $[0, 1]$\footnote{
				See \myRef{PRP: cont funcs on compact are unif cont}.
				}).
			Then $\id$ is uniformly continuous in both directions and yet the metrics are not uniformly equivalent.
			
			\item Let $f$ not be uniformly continuous and $X$ be complete\footnote{
				Note that $X_\text{old}$ is complete iff $X_\text{new}$ is.
				}
			(like $x\to x^3$ on $\mathbb R$). Then Cauchy sequences are just convergent sequences, which are thus the same. However, $\id\colon X_\text{old}\to X_\text{new}$ is not uniformly continuous.
			
			\item Consider $x\mapsto 1/x$ on $\mathbb R^+$. However, note that $1, 2, 3, \ldots$ is Cauchy in the new metric and not in the old one.
		\end{mylist}
	\end{rmk}
	
	\begin{dgrs}
		\begin{lem}\label{LEM: to show counters for equiv of metrics}
			Let $f\colon X\to X$ be a bijection which thus induces a new metric on $X$.\footnote{
				Only injectivity is required if we just need to have a new metric.
				}
			Then \tfh:
			\begin{mylist}
				\item The new metric is topologically equivalent to the old one $\iff$ $f$ is a \homeo on $X_\text{old}$.
				
				\item $\id\colon X_\text{old}\to X_\text{new}$ is uniformly continuous $\iff$ $f$ is uniformly continuous on $X_\text{old}$.
			\end{mylist}
		\end{lem}
		
		\begin{dgrsProof}
			\begin{mylist}
				\item Use \myRef{COR: first countable T1 topologgies via their sequences} and \myRef{PRP: continuity and closure in first countable}.
				
				\item Since by definition, $d_\text{new}(x, y) = d_\text{old}(f(x), f(y))$.\qedhere
			\end{mylist}
		\end{dgrsProof}
	\end{dgrs}



\subsection{Stronger forms of continuity}

	\begin{prp}
		\leavevmode
		\begin{mylist}
			\item Uniform continuity on every totally bounded subset of the domain $\iff$ Cauchy-regularity.
			
			\item Cauchy-regularity $\implies$ continuity.
		\end{mylist}
	\end{prp}
	
	\begin{proof}
		\begin{mylist}
			\item ``$\Rightarrow$'' is obvious since Cauchy sequences are totally bounded.
			
			``$\Leftarrow$'': Suppose $f$ is Cauchy-regular and yet not uniformly continuous on a totally bounded subset $E$ of the domain, so that we may take an $\epsilon > 0$ and for each $n$, choose\myMargin{
				\CC used!
			}
			$x_n, y_n\in E$ such that $d(x_n, y_n) < 1/n$ and yet $d(f(x_n), f(y_n)) \ge \epsilon$. \Wlogg, let $(x_n)$, $(y_n)$ be Cauchy (for $E$ is \ul{totally bounded}). Now, the sequence $x_1, y_1, x_2, y_2, \ldots$ is also Cauchy, and despite that, its $f$-image isn't.
			
			
			\item Let $f$ be Cauchy-regular and $x_i\to c$ in the domain. Then $x_1, c, x_2, c, \ldots$ is Cauchy $\wimplies$ $f(x_1), f(c), f(x_2), f(c), \ldots$ is Cauchy $\wimplies$ $f(x_i)\to f(c)$.\qedhere
		\end{mylist}
	\end{proof}
	
	\begin{rmk}
		\begin{rmklist}
			\item $x\mapsto x^2$ on $\mathbb R$ is Cauchy-regular and not uniformly continuous.
			
			\item $x\mapsto 1/x$ on $\mathbb R^+$ is continuous but not Cauchy-regular.
		\end{rmklist}
	\end{rmk}
	
	
	\begin{thm}[Extension of Cauchy-regulars]\label{THM: extension of Chauchy-regs}
		A Cauchy-regular function from a dense subset to a complete codomain has a unique continuous extension to the whole of domain, which is further Cauchy-regular. Furthermore, this extension preserves uniform continuity and isometry-city.
	\end{thm}
	
	\begin{proof}
		Let $f\colon A\to Y$ be Cauchy-regular where $A$ is dense in $X$, and $Y$ complete. Let's first settle uniquness.\footnote{Which also directly follows from \myRef{PRP: cont func into Hausdorff uniquely by its vals on dense subset}.} Let $x\in X$. Then take a sequence\myMargin{
			\CC used; avoidable if $X$ separable.
		} $(a_i)\in A$ such that $a_i\to x$ (since \ul{$A$ dense in $X$}). If $\tilde f\colon X\to Y$ is a continuous extension of $f$, then we must have $f(a_i)\to \tilde f(x)$.
		
		Let's verify that this indeed gives a well-defined Cauchy-regular extension:
		\begin{prooflist}
			\item Well-defined:
			\begin{mylist}
				\item If $(a_i)$ is Cauchy in $A$, then by \ul{Cauchy-regularity}, it's $f$-image is also Cauchy, and thus \cgt due to \ul{completeness of $Y$}.
				
				\item Let $(a_i), (b_i)\in A$ converge to the same point in $X$. Then the interleaved sequence $a_0, b_0, a_1, b_1, \ldots$ is Cauchy. Due to \ul{Cauchy-regularity}, its $f$-image is also Cauchy $\wimplies$ $\lim_i f(a_i) = \lim_i f(b_i)$.
			\end{mylist}
			
			
			
			\item Extension:
			This is clear since for $a\in A$, the constant sequence $(a, a, \ldots)$ converges to $a$ so that $\tilde f(a) = \lim_i f(a) = f(a)$.
			
			
			
			\item Cauchy-regularity:
			Let $(x^{(n)})\in X$ be Cauchy. We need to show that it's $\tilde f$-image is Cauchy as well. As before due to \ul{denseness of $A$}, choose Cauchy sequences\myMargin{
				\CC used twice; both avoidable if $X$ separable.
			} $( a^{(n)}_i )\in A$ such that $a^{(n)}_i\to x^{(n)}$ so that $\tilde f(x^{(n)}) = \lim_i b^{(n)}_i$. where $b^{(n)}_i := f(a^{(n)}_i)$. Note that
			\begin{mylist}
				\item the Cauchy-ness of $(x^{(n)})$ translates to $\lim_i d(a^{(m)}_i, a^{(n)}_i)\to 0$ as $m, n\to\infty$, and similarly,
				
				\item that of $(\tilde f(x^{(n)}))$ translates to $\lim_i d(b^{(m)}_i, b^{(n)}_i)\to 0$ as $m, n\to\infty$.
			\end{mylist}
			%
			Since the sequences are Cauchy, assume for all $n$'s \wlogg, that $d(a^{(n)}_j, a^{(n)}_i), d(b^{(n)}_j, b^{(n)}_i) < 1/i$ whenever $j\ge i$.
			
			Note that it suffices to get hold of a ``diagonal'' sequence $(b^{(n)}_{N_n})$ that is Cauchy with $N_n$'s increasing\footnote{
				Actually, what is required in the proof is just that $1/N_n\to 0$.
			} for then we'll have
			\begin{alignat*}{2}
				d(b^{(m)}_i, b^{(n)}_i)
				& \le d(b^{(m)}_i, b^{(m)}_{N_m}) + d(b^{(m)}_{N_m}, b^{(n)}_{N_n}) + d(b^{(n)}_{N_n}, b^{(n)}_i)\\
				& < 1/N_m + d(b^{(m)}_{N_m}, b^{(n)}_{N_n}) + 1/N_n & \text{(taking $i\ge N_m, N_n$)}
			\end{alignat*}
			so that we'll have $\lim_i d(b^{(m)}_i, b^{(n)}_i)$ being less than the \RHS which indeed goes to $0$ as $m, n\to \infty$ (since \ul{$(b^{(n)}_{N_n})$ Cauchy} and \ul{$N_n$'s increasing}).
			
			Since \ul{$f$ is Cauchy-regular}, it suffices to find a Cauchy $(a^{(n)}_{N_n})$. Choose $N_n$'s increasing, such that $d(a^{(n)}_i, a^{(n)}_n) < 1/n$ for each $i\ge n$. Now,
			\begin{alignat*}{2}
				d(a^{(m)}_{N_m}, a^{(n)}_{N_n})
				& \le d(a^{(m)}_{N_m}, a^{(m)}_{N_n}) + d(a^{(m)}_{N_m}, a^{(n)}_{N_n})\\
				& < 1/N_m + d(a^{(m)}_{N_m}, a^{(m)}_i) + d(a^{(m)}_i, a^{(n)}_i) & \text{(taking $n\ge m$)}\\
				& \phantom{{} < 1/N_m}\hspace{10pt} {} + d(a^{(n)}_i, a^{(n)}_{N_n})\\
				& < 2/N_m + d(a^{(m)}_i, a^{(n)}_i) + 1/N_n & \text{(taking $i\ge N_m, N_n$)}\\
				& \le 2/N_m + 1/N_n + \lim\nolimits_i d(a^{(m)}_i, a^{(n)}_i) & \text{(taking $i\to \infty$)}
			\end{alignat*}
			which indeed goes to $0$ as $m, n\to\infty$.
		\end{prooflist}
			
		Finally, we verify the preservations:
			\begin{prooflist}
				\item Preservation of uniform continuity:
				Let $f$ be uniformly continuous. We need to show that $\tilde f$ is also uniformly continuous. Let $\epsilon > 0$ and take $\delta > 0$ such that $d(f(a), f(b)) < \epsilon$ whenever $d(a, b) < \delta$ for $a, b\in A$. Now, let $x, y\in X$ with $d(x, y) < \delta$. Take\myMargin{
					Same comment on \CC.
				} $(a_i), (b_i)\in A$ converging to $x$, $y$ \resp. Now, $d(a_i, b_i) < \delta$ eventually (as $\lim_i d(a_i, b_i) = d(x, y) < \delta$) so that $d(f(a_i), f(b_i)) < \epsilon$ eventually $\wimplies$ $d(\tilde f(x), \tilde f(y)) \which= \lim_i d(f(a_i), f(b_i))\le \epsilon$.
				
				
				\item Preservation of isometry-city:
				Same technique as in the last point.
				\qedhere
			\end{prooflist}
	\end{proof}
	
	Thus, for a continuous function to be Cauchy-regular, it must be continuously extensible to the completion of its domain.
	
	\begin{cor}
		If $A$ is dense in $X$ and $\iota\colon X\to \hat X$ a completion of $X$, then
		\[
		\begin{tikzcd}
			A\ar[r, hook] & X\ar[r, "\iota"] & \hat X
		\end{tikzcd}
		\]
		is a completion of $A$.
	\end{cor}
	
	\begin{proof}
		It's clear that it's an isometry. Now, for $Y$ complete any isometry $f\colon A\to Y$ extends to $X\to Y$ via \myRef{THM: extension of Chauchy-regs} (since \ul{$A$ dense in $X$}), which then extends to $\hat X\to Y$:
		\[
		\begin{tikzcd}
			A\ar[r, hook]\ar[dr, "f"'] & X\ar[r, "\iota"]\ar[d, dashed] & \hat X\ar[dl, dashed]\\
			& Y &
		\end{tikzcd}
		\]
		
		For uniqueness, let $f$ be extended by $\tilde f_1$ and $\tilde f_2$:
		\[
		\begin{tikzcd}
			A\ar[r, hook]\ar[dr, "f"'] & X\ar[r, "\iota"] & \hat X\ar[dl, shift left, "\tilde f_1"']\ar[dl, shift right, "f_2"]\\
			& Y &
		\end{tikzcd}
		\]
		Then $\tilde f_i\circ \iota$'s are continuous extensions of $f$. Since \ul{$A$ is dense in $X$}, these must be equal due to \myRef{PRP: cont func into Hausdorff uniquely by its vals on dense subset}, from where there equality follows from the universal property of the completion $\iota\colon X\to\hat X$.
	\end{proof}
	
	\noindent This immediately yields:
	
	\begin{cor}
		In a complete space, the closures of subsets are their completions.
	\end{cor}
	
	

	\begin{prp}\label{PRP: cont funcs on compact are unif cont}
		Continuous functions on compact sets are uniformly continuous.
	\end{prp}
	
	\begin{proof}
		Let $f\colon X\to Y$ be continuous with $X$ being compact. Let $\epsilon > 0$. For each $x\in X$, choose $\delta_x > 0$ such that $f(B_{\delta_x}(x))\subseteq B_\epsilon(f(x))$.\myMargin{No \AC needed!}
		Let $B_{\delta_{x_1}/2}(x_1), \ldots, B_{\delta_{x_n}/2}(x_n)$ cover $X$ (since \ul{$X$ is compact}). Now, any $x, y\in X$ lie in some $B_{\delta_{x_i}}(x_i)$ whenever $d(x, y) < \min(\delta_{x_1}, \ldots, \delta_{x_n})/2\wimplies d(f(x), f(y)) < 2\epsilon$.
	\end{proof}
	
	\begin{rmk}
		To see the necessity of compact domain, consider $x\mapsto 1/x$ on $\mathbb R\setminus\{0\}$.
	\end{rmk}
	
	
	\begin{prp}[Uniform convergence preserves continuity]
		Let $E$ be a topological space and $E_1\subseteq E$. Let $f_n\colon E_1\to X$ converge uniformly to $f$. Let $c\in\ell(E_1)$ with each $\lim_{x\to c} f_n(x)$ existent. Then
		\[
		\lim_{x\to c} f(x) = \lim_{n\to\infty}\lim_{x\to c}f_n(x)
		\]
		for each $c\in\ell(E_1)$.\footnote{Note that metric spaces are Hausdorff so that limits are unique.}
	\end{prp}
	
	\begin{proof}
		
	\end{proof}
	
	\begin{rmk}
		content
	\end{rmk}
	
	
	


\section{Baire's Category Theorem}

	\begin{prp}[Cantor's intersection]
		In a complete metric space, the intersection of a decreasing sequence of closed subsets with diameters going to zero, is a singleton.
	\end{prp}
	
	\begin{proof}
		Let $F_i$'s be the closed sets under consideration. That there's at most one point in the intersection is clear since \ul{$\delta(F_i)\to 0$}. Now, choose\myMargin{\CC used.}
		$x_i\in F_i$, which form a Cauchy sequence since \ul{$\delta(F_i)\to 0$}. Since the space is \ul{complete}, let $x_i\to x$, and since \ul{each $F_i$ is closed}, $x$ lies in the intersection.
	\end{proof}
	
	\begin{rmk}
		The necessity of each hypothesis is easy to see.
	\end{rmk}
	
	\begin{dgrs}
		The diameter of the intersection of a decreasing sequence of subsets needn't be the corresponding limit of diameters even if the sets are closed and bounded. For instance, consider an infinite dimensional \NLS containing orthonormal vectors $e_1, e_2, \ldots$. Take $F_i := \{e_i, e_{i + 1}, \ldots\}$. Then each $\delta(F_i) = \sqrt 2$, and still the intersection is empty. However, there is one case where we \emph{can} say something:
		
		\begin{prp}
			Let $F_1\supseteq F_2\supseteq\cdots$ be closed subsets of a metric space with $F_1$ being compact. Then $\delta(\bigcap_i F_i) = \lim_i\delta(F_i)$.
		\end{prp}
		
		% COMPACTNESS => SEQUENTIAL COMPACTNESS
		\begin{dgrsProof}
			``$\le$'' is clear. For ``$\ge$'', let $\epsilon> 0$ and choose $x_i, y_i\in F_i$ such that $d(x_i, y_i) > \delta(F_i) - \epsilon$ (note that each $\delta(F_i) < +\infty$). Now, since \ul{$F_1$ is compact}, let $x_{n_i}\to x$ and $y_{n_i}\to y$ in $F_1$. Since \ul{$F_i$'s are closed}, $x$, $y$ lie in the intersection so that $\delta(\bigcap_i F_i)\ge d(x, y)\ge \lim_i\delta(F_i) - \epsilon$.
		\end{dgrsProof}
	\end{dgrs}
	
	\begin{thm}[Baire's category]\label{THM: BCT}
		In a complete metric space, complements of meager sets are dense.
	\end{thm}
	
	\begin{proof}
		Let $A_1, A_2, \ldots$ be nowhere dense. We show that $X\setminus \bigcup_i A_i$ is dense. Pick a nonempty open $U$. Since \ul{$A_1$ is nowhere dense}, choose $x_1\in U$ and $r_1 > 0$ such that $B_{r_1}(x_1)\subseteq U$ and $B_{r_1}(x_1)\cap A_1 = \emptyset$. Having chosen\myMargin{\DC used.}
		$x_i$, $r_i$, choose $x_{i + 1}\in B_{r_i}(x_i)$ such that
		\begin{assmplist}
			\item $B_{r_{i + 1}}(x_{i + 1})\subseteq B_{r_i}(x_i)$,
			\item $r_{i + 1}\le r_i/2$, and
			\item $B_{r_{i + 1}}(x_{i + 1})\cap A_{i + 1} = \emptyset$.
		\end{assmplist}
		This is possible since \ul{$A_i$ is nowhere dense}.
		Thus, $\clos{B_{r_1}(x_1)}\supseteq\clos{B_{r_2}(x_2)}\supseteq\cdots$ with $\delta(\clos{B_{r_i}(x_i)})\le\delta(D_{2r_i}(x_i)) \which = 2r_i\which\to 0$ since $r_i\le r_1/2^{i - 1}$. By Cantor (since \ul{$X$ is complete}), let $x\in\bigcap_i\clos{B_{r_i}(x_i)}$. Then $x\notin \bigcup_i A_i$ since each $B_{r_i}(x_i)\cap A_i = \emptyset$. Finally, $x\in \clos{B_{r_1}(x_1)}$ and \wlogg, we cloud've chosen $x_1$, $r_1$ such that $\clos{B_{r_1}(x_1)}\subseteq U$.
	\end{proof}
	
	\begin{rmk}
		Necessity of completeness is demonstrated by considering any countable metric space in which singletons are not open, for instance $\mathbb Q$.
	\end{rmk}
	
	\begin{cor}
		If countably many closed subsets of a nonempty complete metric space unite to the whole space, then one of them has a nonemtpy interior.
	\end{cor}