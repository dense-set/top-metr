%=========Checkpoints=========%
%
% COMPACTNESS => SEQUENTIAL COMPACTNESS
% p-norms are norms

\chapter{Metric Spaces}

\begin{conv}
	Unless stated otherwise, assume the following:
	\begin{assmplist}
		\item $X$, $Y$ will denote metric spaces.
		
		\item $E$ will be reserved for generic sets.
		
		\item Subsets of metric spaces will be seen as metric subspaces.
		
		\item For $x\in X$ and $r > 0$, we'll use
		\begin{assmplist}
			\item $B(x, r) := \{y\in X : d(y, x) < r\}$, and
			\item $D(x, r) := \{y\in X : d(y, x)\le r\}$.
		\end{assmplist}
		Sometimes, we'll also denote these by $B_r(x)$ and $D_r(x)$.
		
		\item The diameter of a subset $A$ of a metric space will be denoted by $\delta(A)$.
		
		\item For any $f, g\in X^E$, we'll define $d_\infty(f, g) := \sup_{e\in E}d(f(e), g(e))$.
		
		\item A metric space will also be considered a topological space under the induced topology.
		
		\item Finite product of metric spaces will be considered together with the metric generated by any of the $p$-norms (which are uniformly equivalent\footnote{
			$\norm{x}_\infty\le \norm{x}_p\le n^{1/p}\norm{x}_\infty$ for $x\in K^n$.
		}). (See \myRef{PRP: finite prod of metr spaces is metrizable}.)
	\end{assmplist}
\end{conv}

\section{General}
	
	\begin{lem}
		\begin{mylist}
			\item Metric is Lipschitz continuous. 
			
			\item If $E\subseteq X$ is nonempty, then $x\mapsto d(x, E)$ is also Lipschitz continuous.\footnote{
				For Lipschitz continuity, the codomain space must be a metric space and thus we must not include $+\infty$ in the codomain. Taking $E\ne\emptyset$ ensure this.
			}
		\end{mylist}
	\end{lem}
	
	\begin{proof}
		\begin{mylist}
			\item Use the triangle inequality with the $1$-norm metric on $X\times X$. (See \myRef{PRP: finite prod of metr spaces is metrizable}.)
			
			\item For any $e\in E$, we have $d(x, e)\le d(x, y) + d(y, e)$ so that taking infimum over $e$ yields $d(x, E)\le d(x, y) + d(y, E)$\myMargin{
				Do stuff on $\mathbb R^*$. Also that bit on monotone functions and taking $\sup$'s.
			} $\wimplies d(x, E) - d(y, E)\le d(x, y)$ (note that $d(x, E), d(y, E) < +\infty$ since \uline{$E\ne \emptyset$}).\qedhere
		\end{mylist}
	\end{proof}
	
	\begin{rmk}
		Note that $d(x, \emptyset) = +\infty$ for all $x$.
	\end{rmk}
	
	
	\begin{lem}\label{LEM: separable metric spaces are second countable}
		\leavevmode
		\begin{mylist}
			\item Metric spaces are first countable.
			\item Separable metric spaces are second countable.
		\end{mylist}
	\end{lem}
	
	\begin{proof}
		\begin{mylist}
			\item $B_{1/n}(x)$'s forms a local base at $x$.
			
			\item Let $S$ be a countable dense subset of $X$. Then $\bigcup_{x\in S}\{ B_{1/n}(x) : n\ge 1 \}$ forms a countable base:
			\begin{subproof}
				Consider $B_{1/n}(y)$. Let $x\in B_{1/2n}(y)\cap S$. Then $y\in B_{1/2n}(x)\subseteq B_{1/n}(y)$.\qedhere
			\end{subproof}
		\end{mylist}
	\end{proof}
	
	
	\begin{lem}
		The subspace topology on a subset of a metric space is precisely the one induced by the inherited metric.
	\end{lem}
	
	\begin{proof}
		Let $E\subseteq X$. Let $\tau_s$, $\tau_m$ be the respective topologies. That $\tau_m\subseteq \tau_s$ is clear since the balls of $E$ are precisely the balls of $X$ intersected with $E$. We show $\tau_s\subseteq \tau_m$:
		\begin{subproof}
			Consider $B_r(x)\cap E$ for $x\in X$. Let $y\in B_r(x)\cap E$. Take $B_\epsilon(y)\subseteq B_r(x)\wimplies y\in B_\epsilon(y)\cap E\subseteq B_r(x)\cap E$.\qedhere
		\end{subproof}
	\end{proof}
	
	Let $E\subseteq X$ and $x\in X$. Then a point $y\in E$ is called \defn{a point of best approximation} for $x$ in $Y$ iff $d(x, y) = d(x, E)$.
	
	
	
	
\section{Products of Metric Spaces}

	We'll try to gain some insight into the following question in this section: Given metric spaces $X_i$'s, is there a metric on $\prod X_i$ that induces the product topology on $\prod X_i$?
	
	\begin{prp}[Finite products]\label{PRP: finite prod of metr spaces is metrizable}
		If $X_1, \ldots, X_n$ are metric spaces and $\norm\cdot$ a norm on $\mathbb R^n$ which is monotonic along each cardinal direction at each point in the orthant $[0, +\infty)^n$, then
		\[
		d(x, y) := \left\Vert \bigl(d_1(x_1, y_1), \ldots, d_n(x_n, y_n)\bigr) \right\Vert\addtag\label{EQN: metric on finite product}
		\]
		defines a metric on $X_1\times\cdots\times X_n$ that induces the product topology on it. Further:
		\begin{mylist}
			\item\label{PRPi: finite prod of metr spaces is metrizable} Cauchy-ness of a sequence in the product is equivalent to the Cauchy-ness of the component sequences in product spaces.
			
			\item\label{PRPii: finite prod of metr spaces is metrizable} If all the spaces are nonempty, then $\prod_i X_i$ is complete $\iff$ each $X_i$ is.
		\end{mylist}
	\end{prp}
	
	\begin{proof}
		That it's a metric is easily verified:
		\begin{prooflist}
			\item $d(x, y) = 0$ $\iff$ each $d_i(x_i, y_i) = 0$ (since \uline{$\norm\cdot$ is positive definite}) $\iff$ each $x_i = y_i$ (since \uline{$d_i$'s are positive definite}) $\iff$ $x = y$.
			
			\item $d$ is symmetric since \uline{each $d_i$ is}.
			
			\item $d$ satisfies triangle inequality since \uline{$\norm\cdot$ and all $d_i$'s do} and \uline{$\norm\cdot$ is monotonic along the cardinal directions in the orthant}.
		\end{prooflist}
		
		We now verify that these induce the product topology. Because the norms on $\mathbb R^n$ are uniformly equivalent (\myRef{COR: unif equiv of norms on fdvs}) and uniformly equivalent metrics are topologically equivalent, we may assume \wlogg that $\norm\cdot$, \uline{which is a norm}, is the $\max$-norm, which clearly generates the product topology.
		
		For \ref{PRPi: finite prod of metr spaces is metrizable}, because Cauchy-ness of a sequence is preserved under uniformly equivalent metrics, we may again work with $\max$-norm which makes the statement obvious.
		
		\ref{PRPii: finite prod of metr spaces is metrizable} follows immediately from \ref{PRPi: finite prod of metr spaces is metrizable} and the characterization of convergence in the product topology (\myRef{PRP: convergence in prod topo}).
	\end{proof}
	
	\begin{rmk}
		Note that to conclude that $d$ is a metric, full power of $\norm\cdot$ being a norm was not used---just that it's positive definite, and that it satisfies triangle inequality and the monotonicity assumption. However, the fact that it's a norm was used in concluding that it generates the product topology.
		
		Secondly, to see the necessity of monotonicity of $\norm\cdot$, consider the following digression:
		\begin{dgrs}
			Any linear injection $T$ on a vector space over $K$ into itself gives a means to produce a new norm from any given norm on it, given by $\norm{x}_\text{new} := \norm{Tx}_\text{old}$.
			
			Thus, consider the linear isomorphism $T$ on $\mathbb R^2$ given by $x\mapsto 
			\begin{bsmallmatrix*}[r]
				a & a\\
				b & -b
			\end{bsmallmatrix*}\, x$
			where $a, b > 0$, and let $\norm\cdot$ be the norm that $T$ generates out of $\norm{\cdot}_1$.\myMargin{
				Add a diagram showing the unit ball of the new norm.
			}
			We then show that $d$ as defined by \myRef{EQN: metric on finite product} needn't satisfy the triangle inequality. Indeed, for $x, y\in X_1\times X_2$, we have
			\begin{align*}
				d(x, y) 
				& = \left\lVert \bigl( d_1(x_1, y_1), d_2(x_2, y_2) \bigr) \right\lVert\\
				& = \left\lVert T\bigl( d_1(x_1, y_1), d_2(x_2, y_2) \bigr) \right\lVert_1\\
				& = a\,\bigl(d_1(x_1, y_1) + d_2(x_2, y_2)\bigr)
				+ b\,\bigl| d_1(x_1, y_1) - d_2(x_2, y_2) \bigr|\text.
			\end{align*}
			Take $x, y, z\in X_1\times X_2$ such that $d_i(x_i, y_i) = 1 = d_i(y_i, z_i)$ for $i = 1, 2$. Set $\alpha_i := d_i(x_i, z_i)$. Then $d(x, y) + d(y, z) = 4a$ and $d(x, z) = a\,(\alpha_1 + \alpha_2) + b\,\abs{\alpha_1 - \alpha_2}\ge b\, \abs{\alpha_1 - \alpha_2}$. Thus, ensuring $4a/b < \abs{\alpha_1 - \alpha_2}$ ensures the violation of triangle inequality.
		\end{dgrs}
	\end{rmk}
	
	
	\begin{prp}\label{PRP: countable prd's of metr spaces are metrizable}
		On countable product of metric spaces is metrizable such that the analogues of \ref{PRPi: finite prod of metr spaces is metrizable} and \ref{PRPii: finite prod of metr spaces is metrizable} of \myRef{PRP: finite prod of metr spaces is metrizable} hold.
	\end{prp}
	
	\begin{proof}
		The finite case follows from \myRef{PRP: finite prod of metr spaces is metrizable}. Thus, consider the metric spaces $X_1, X_2, \ldots$, and set $X := \prod_i X_i$. Because of \myRef{PRP: any metric is topologically equiv to a bdd metric}, we may \wlogg assume that each $d_i$ is bounded by $1$ which allows to define
		\[
		d(x, y) := \sum_{i = 1}^\infty \frac{d_i(x_i, y_i)}{2^i}
		\]
		for $x, y\in X$. That $d$ defines a metric is immediate. We show the topological equivalence:
		\begin{prooflist}
			\item Consider a generic set, $\pi_i^{-1}(B^{(i)}_r(x_i))$ with $x_i\in X_i$\footnote{
				Yes, notation's being abused.
			} for a fixed $i$, that is united over to form a subbasic set of the product topology. \Wlogg, assume that $X\ne \emptyset$, so that we may take an $x\in X$ such that it's $i$-th coordinate is the $x_i$ above. It suffices to find an $\epsilon > 0$ such that $B_\epsilon(x)\subseteq \pi_i^{-1}( B^{(i)}_r(x_i) )\wimpliedby \pi_i(B_\epsilon(x))\subseteq B^{(i)}_r(x_i)$. Now, let $y_i\in \LHS$\footnote{
				Again abusing notation.
			} so that there's a $y\in B_\epsilon(x)$ whose $i$-th coordinate is $y_i$. Then $d_i(y_i, x_i)/2^i\le d(y, x) < \epsilon\wimplies d_i(y_i, x_i) < 2^i \epsilon$. Thus it suffices to have $\epsilon < r/2^i$ to ensure that $y_i\in B^{(i)}_r(x_i)$.
			
			\item Consider a basic open set $B_r(x)$ of the metric topology on $X$. It suffices to find $\epsilon > 0$ and $x_j\in X_j$ for finitely many $j$'s such that $\bigcap_j \pi_j^{-1}(B^{(j)}_\epsilon(x_j))\subseteq B_r(x)$. Let $j$'s come from $\{1, \ldots, n\}$ and $y\in \bigcap_j \pi_j^{-1}(B^{(j)}_\epsilon(x_j))$ so that each $d_j(y_j, x_j) < \epsilon$. Thus, since \uline{$d_i$'s are bounded by $1$}, we have $d(y, x) < \epsilon + 1/2^n\which < r$ if $1/2^n < r - \epsilon$, which can be ensured by taking $\epsilon < r$ and $n$ large enough.
		\end{prooflist}
		
		We now show \ref{PRPi: finite prod of metr spaces is metrizable}, from which \ref{PRPii: finite prod of metr spaces is metrizable} follows immediately:
		\begin{subproof}
			Let $(x^{(n)})\in X$ be Cauchy. Fix an $i$. Then $d_i(x^{(m)}_i, x^{(n)}_i)/2^i\le d(x^{(m)}, x^{(n)})\to 0$ as $m, n\to\infty$.
			
			Conversely, let $(x^{(n)}_i)_n\in X_i$ be Cauchy for each $i$. Let $\epsilon > 0$. Fix an $N$. Then $d(x^{(m)}, x^{(n)}) = \sum_{i\le N} d_i(x^{(m)}_i, x^{(n)}_i)/2^i + 1/2^N\which < \epsilon$ if $\sum_{i\le N} d_i(x^{(m)}_i, x^{(n)}_i)/2^i < \epsilon - 1/2^N$. Thus, take $N$ such that $\epsilon - 1/2^N > 0$ and then use that Cauchy-ness of $(x^{(n)}_i)_n$ for $i\le N$.
		\end{subproof}
	\end{proof}
	
	\begin{rmk}
		To see the necessity of countability, consider an uncountable product of discrete $\{0, 1\}$, which is not first countable and hence not metrizable.
	\end{rmk}
	

	
	

\section{Uniform Properties}

Uniform properties encompass things like uniform continuity and Cauchy sequences.


	\begin{lem}\label{LEM: closed subsets in complete spaces}
		In a complete space, closed subspaces are precisely the complete ones.
	\end{lem}
	
	
	\begin{lem}\label{LEM: totally bdd iff each seq has a Cauchy subseq}
		In a metric space, each sequence has a Cauchy subsequence $\iff$ the space is totally bounded.
	\end{lem}
	
	\begin{proof}
		``$\Rightarrow$'': Consider a sequence $(x_i)$. Take an infinite subset $I_1$ of the indices such that $\{x_i : i\in I_1\}$ lies in a ball of diameter $1$. Having chosen $I_n$, choose\myMargin{
			\DC used.}
		an infinite subset $I_{n + 1}\subseteq I_n$ such that $\{x_i : i\in I_{n + 1}\}$ lies in a ball of diameter $1/(n + 1)$. (This is possible since the the space is \uline{totally bounded}.) Now, choose $i_n\in I_n$ such that $(i_n)$ is increasing. Then $(x_{i_n})_n$ forms a Cauchy sequence, for for $n > m\ge N$, we have $d(x_{i_m}, x_{i_n}) < 1/N$.
		
		``$\Leftarrow$'': Suppose $X$ is not totally bounded so that take an $\epsilon > 0$ such that no finitely many balls of radius $\epsilon$ can ever cover $X$. Let $x_1\in X$.\footnote{
			Note that $X$ has got to be nonenmtpy.}
		Having chosen $x_1, \ldots, x_n$, choose\myMargin{
			\DC used.}
		$x_{n + 1}\in X\setminus\bigcup_{i = 1}^n B_\epsilon(x_i)$. Then $(x_i)$ is non-Cauchy sequence, for $d(x_i, x_j)\ge \epsilon$ for all $i\ne j$.
	\end{proof}
	
	\begin{rmk}
		The discrete metric on an infinite set, which is not totally bounded, contains sequences with no Cauchy subsequences.
	\end{rmk}



\subsection{Completion of metric spaces}

	A \defn{completion} of $X$ is a complete metric space $\hat X$ together with an isometry $\iota\colon X\to \hat X$ such that any isometry $X\to Y$ into a complete metric space $Y$ factors uniquely through $\iota$ via an isometry:
	\[
	\begin{tikzcd}[column sep=small]
		X\ar[rr]\ar[dr, "\iota"'] & & Y\\
		& \hat X\ar[ur, dashed] & 
	\end{tikzcd}
	\]
	
	\begin{cor}
		If $X$ is complete, then $\id\colon X\to X$ is a completion of $X$.
	\end{cor}
	
	\begin{thm}\label{THM: completion of metric spaces}
		Each metric space admits a completion, which is unique up to bi-isometries.\footnote{
			A bi-isometry is a bijective isometry whose inverse is also an isometry. Thus, bi-isometries are precisely bijective isometries.
		}
	\end{thm}
	
	\noindent The following easy facts will be employed to simplify the proof:
	
	\begin{lem}\label{LEM: some properties of Cauchy sequences}
		\leavevmode
		\begin{mylist}
			\item If $(x_i)$ is Cauchy and $(\alpha_i)\in\mathbb R^+$, then there exists a subsequence $(x_{i_j})$ such that for each $N$, we have $d(x_{i_j}, x_{i_k}) < \alpha_N$ whenever $j, k\ge N$.
			
			\item A Cauchy sequence converges $\iff$ any of its subsequence converges.
		\end{mylist}
	\end{lem}
	
	\begin{proof}[Proof of \myRef{THM: completion of metric spaces}]
		The uniqueness follows by the usual categorical argument. Let's show the existence of a completion of $X$. Define $\hat X$ to be set of the Cauchy sequences in $X$ modded out by the following equivalence relation:
		\[
		(x_i)\sim (y_i) \text{ iff } d(x_i, y_i)\to 0
		\]
		The following defines a well-defined metric on $\hat X$:
		\[
		\hat d\bigl( \overline{(x_i)}, \overline{(y_i)} \bigr) := \lim\nolimits_i d(x_i, y_i)
		\]
		We show that $\hat X$ is complete:
		\begin{subproof}
			Let $\bigl(\,\overline{x^{(n)}}\,\bigr)$ be Cauchy in $\hat X$, where each $x^{(n)}$ is a Cauchy sequence $\bigl(x^{(n)}_i\bigr)$ in $X$.\myMargin{
				\CC used.
			} Noting that each subsequence of a Cauchy sequence in $X$ is related to the parent sequence, and due to \myRef{LEM: some properties of Cauchy sequences}, we may \wlogg assume:
			\begin{mylist}
				\item $n \ge m\implies \hat d\bigl(\, \overline{x^{(n)}}, \overline{x^{(m)}}\, \bigr) < 1/m$.
				
				\item For each $n$, we have $j\ge i\implies d\bigl(x^{(n)}_j, x^{(n)}_i\bigr) < 1/i$.
			\end{mylist}
			Now, it follows that the diagonal sequence $\bigl(x^{(i)}_i\bigr)$ is Cauchy:
			\begin{alignat*}{2}
				d\bigl(x^{(j)}_j, x^{(i)}_i\bigr)
				& \le d\bigl(x^{(j)}_j, x^{(i)}_j\bigr) + d\bigl(x^{(i)}_j, x^{(i)}_i\bigr)\\
				& < d\bigl(x^{(j)}_j, x^{(j)}_k\bigr) + d\bigl(x^{(j)}_k, x^{(i)}_k\bigr) + d\bigl(x^{(i)}_k, x^{(i)}_j\bigr) & \text{($k$ arbitrary)}\\
				& \phantom{{} < d\bigl(x^{(j)}_j, x^{(j)}_k\bigr) + d\bigl(x^{(j)}_k, x^{(i)}_k\bigr) abc} {}+ 1/i & \text{(letting $j\ge i$)}\\
				& < 1/j + d\bigl(x^{(j)}_k, x^{(i)}_k\bigr) + 1/j +1/i & \text{(letting $k\ge j$)}\\
				& \le 2/i + \hat d\bigl(\, \overline{x^{(j)}},\, \overline{x^{(i)}}\, \bigr) + 1/i & \text{(taking $k\to \infty$)}\\
				& < 2/j + 2/i & \text{(since $j\ge i$)}
 			\end{alignat*}
			Also, $\overline{x^{(n)}}\to \overline{\bigl(x^{(i)}_i\bigr)}$ in $\hat X$:
			\begin{alignat*}{2}
				d\bigl( x^{(n)}_i, x^{(i)}_i \bigr)
				& \le d\bigl( x^{(n)}_i, x^{(n)}_j \bigr) + d\bigl( x^{(n)}_j, x^{(i)}_j \bigr) + d\bigl( x^{(i)}_j, x^{(i)}_i \bigr)\\
				& < 1/i + d\bigl( x^{(n)}_j, x^{(i)}_i \bigr) + 1/i & \text{(letting $j\ge i$)}\\
				& < 2/i + d\bigl(\, \overline{x^{(n)}},\,\overline{x^{(i)}} \,\bigr) & \text{(taking $j\to\infty$)}\\
				& < 3/i
			\end{alignat*}
		\end{subproof}

		\noindent We now check for the universal property:
		\begin{subproof}
			Note that $\iota\colon X\to\hat X$ given by $x\mapsto \overline{(x, x, \ldots)}$ is an isometry. Let $f\colon X\to Y$ be another isometry with $Y$ being complete. Suppose it does factor through $\iota$ via an isometry $\hat f$:
			\[
			\begin{tikzcd}[column sep=small]
				X\ar[dr, "\iota"']\ar[rr, "f"] && Y\\
				& \hat X\ar[ur, dashed, "\hat f"'] &
			\end{tikzcd}
			\]
			This in turn determines $\hat f$ uniquely:
			\begin{subproof}
				Let $\overline{(x_i)}\in\hat X$, where $(x_i)$ is Cauchy in $X$. Clearly, $\iota(x_i)\to\overline{(x_i)}$ so that
				\[
				f(x_i)\to \hat f\bigl(\, \overline{(x_i)} \,\bigr) \addtag\label{EQN: metric completion proof}
				\]
				as \uline{$\hat f$ is continuous} and \uline{$\hat f\circ \iota = f$}.
			\end{subproof}
			We now show that \myRef{EQN: metric completion proof} indeed defines a factoring of $f$ via $\iota$:
			\begin{prooflist}
				\item $\hat f$ is well-defined:
				\begin{rmklist}
					\item If $(x_i)$ is Cauchy in $X$, then since \uline{$f$ is an isometry} and \uline{$Y$ is complete}, $(f(x_i))$ is convergent in $Y$.
					
					\item If $(x_i)$ and $(y_i)$ are equivalent Cauchy sequences in $X$, then $d(x_i, y_i)\to 0\wimplies d(f(x_i), f(y_i))\to 0$ (since \uline{$f$ an isometry}) so that $d\bigl(\lim_i f(x_i), \lim_i f(y_i)\bigr) = 0$.\footnote{Recall the component-wise convergence in product topology.}
				\end{rmklist}
				
				\item $\hat f$ is an isometry:
				\begin{alignat*}{2}
					d\Bigl( \hat f\bigl(\, \overline{(x_i)} \,\bigr),\, \hat f\bigl(\, \overline{(y_i)} \,\bigr) \Bigr)
					& = d\bigl(\lim\nolimits_i f(x_i), \lim\nolimits_i f(y_i)\bigr)\\
					& = \lim\nolimits_i d(f(x_i), f(y_i))\\
					& = \lim\nolimits_i d(x_i, y_i) & \text{(\uline{$f$ is an isometry})}\\
					& = \hat d\bigl(\, \overline{(x_i)},\, \overline{(y_i)} \,\bigr)
				\end{alignat*}
				
				\item Finally, $\hat f\circ \iota = f$ is clear.\qedhere
			\end{prooflist}
		\end{subproof}
	\end{proof}
	
	\begin{comment}
		\begin{rmk}
		Notice that while proving the uniqueness of $\hat f$, full power of $\hat f$'s isometry was not used, just the continuity. Thus, we can strengthen the universal property like so: Any $f\colon X\to Y$ with $Y$ complete factors uniquely via $\iota$, say through $\hat f$, and this $\hat f$ is further an isometry.
	\end{rmk}
	\end{comment}
	
	
	\begin{prp}\label{PRP: a space is dense in its completion}
		Any space is dense in its completion.
	\end{prp}
	
	\begin{proof}
		Let $\iota\colon X\to \hat X$ be a completion of $X$. Now, the restriction $\iota|\colon X\to \clos{\iota(X)}$ is also a completion of $\hat X$:
		\begin{prooflist}
			\item $\clos{\iota(X)}$ is complete due to \myRef{LEM: closed subsets in complete spaces} and $\iota|$ is still an isometry.
			
			\item Let $f\colon X\to Y$ be an isometry with $Y$ complete. Then $f$ factors through $\iota$ which induces a factoring through $\iota|$ as well:
			\[
			\begin{tikzcd}
				& X\ar[dl, "\iota|"']\ar[d, "\iota"]\ar[dr, "f"] &\\
				\clos{\iota(X)}\ar[r, hook] & \hat X\ar[r, dashed] & Y
			\end{tikzcd}
			\]
			For uniqueness, just note that any isometric factoring $\tilde f$ of $f$ through $\iota|$ is determined on $\iota(X)$ which is dense in $\clos{\iota(X)}$, thereby also getting determined on $\clos{\iota(X)}$ (by \myRef{PRP: cont func into Hausdorff uniquely by its vals on dense subset}, for $\tilde f$ is continuous):
			\[
			\begin{tikzcd}
				& X\ar[dl]\ar[d, "\iota|"]\ar[dr, "f"] &\\
				\iota(X)\ar[r, hook] & \clos{\iota(X)}\ar[r, "\tilde f"'] & Y
			\end{tikzcd}
			\]
		\end{prooflist}
		
		Since \uline{$\hat X$ is a completion}, there exists an isometry $\alpha$ such that \tfdc:
		\[
		\begin{tikzcd}[column sep=small]
			& X\ar[dl, "\iota|"']\ar[dr, "\iota"] &\\
			\clos{\iota(X)}\ar[rr, hook, "\operatorname{incl}", shift left] & & \hat X\ar[ll, dashed, "\alpha", shift left]
		\end{tikzcd}
		\]
		It follows that $\iota$ factors through itself via $\operatorname{incl}\circ \alpha$, so that it is precisely $\id_{\hat X}$ $\wimplies$ $\operatorname{incl}$ is surjective $\wimplies$ $\clos{\iota(X)} = \hat X$.
	\end{proof}
	
	\begin{rmk}
		It turns out that this in fact characterizes completions. See \myRef{CORii: immediate conseqs of extension of Cauchy-regs} of \myRef{COR: immediate conseqs of extension of Cauchy-regs}.
	\end{rmk}
	
	
	From \myRef{LEM: denseness is transitive}, it now immediately follows that:
	
	\begin{cor}
		Completion preserves separability.
	\end{cor}
	
	
\subsection{Metric equivalences}

	\begin{prp}\label{PRP: implications from unif equiv to topo equi}
		For metrics on a given set, we have:
		\begin{center}
			Uniform equivalence $\implies$ $\id$ is uniformly continuous in both directions $\implies$ same Cauchy sequences $\implies$ same convergence $\iff$ topological equivalence.
		\end{center}
	\end{prp}
	
	\begin{proof}
		The first two implications are trivial and the last follows from \myRef{COR: first countable T1 topologies via their sequences}. For the penultimate, just note that $x_i\to c$ $\iff$ $x_1, c, x_2, c, \ldots$ is Cauchy.
	\end{proof}
	
	\begin{rmk}
		None of the converses are true. Let $f\colon X\to X$ be a \homeo which thus induces a topologically equivalent metric on $X$ (see \myRef{LEM: to show counters for equiv of metrics}).
		\begin{mylist}
			\item Let $f$, $f^{-1}$ be uniformly continuous and $f$ not be Lipschitz (for instance, $f\colon x\mapsto \sqrt x$ on $[0, 1]$\footnote{
				See \myRef{PRP: cont funcs on compact are unif cont}.
				}).
			Then $\id$ is uniformly continuous in both directions and yet the metrics are not uniformly equivalent.
			
			\item Let $f$ not be uniformly continuous and $X$ be complete\footnote{
				Note that $X_\text{old}$ is complete $\iff$ $X_\text{new}$ is.
				}
			(like $x\to x^3$ on $\mathbb R$). Then Cauchy sequences are just convergent sequences, which are thus the same. However, $\id\colon X_\text{old}\to X_\text{new}$ is not uniformly continuous.
			
			\item Consider $f\colon x\mapsto 1/x$ on $X := \mathbb R^+$. However, note that $1, 2, 3, \ldots$ is Cauchy in the new metric and not in the old one.
		\end{mylist}
	\end{rmk}
	
	\begin{dgrs}
		Any injection $f$ on a set into itself gives a means to generate a new metric given any metric on it via $d_\text{new}(x, y) := d_\text{old}(f(x), f(y))$.
		
		\begin{lem}\label{LEM: to show counters for equiv of metrics}
			Let $f\colon X\to X$ be a bijection which thus induces a new metric on $X$. Then \tfh:
			\begin{mylist}
				\item The new metric is topologically equivalent to the old one $\iff$ $f$, $f^{-1}$ are continuous on $X_\text{old}$.
				
				\item $\id\colon X_\text{old}\to X_\text{new}$ is uniformly continuous $\iff$ $f$ is uniformly continuous on $X_\text{old}$.
				
				\item The new metric is uniformly equivalent to the old one $\iff$ $f$, $f^{-1}$ are Lipschitz on $X_\text{old}$.
			\end{mylist}
		\end{lem}
		
		\begin{dgrsProof}
			For the first use \myRef{PRP: continuity and closure in first countable} and \myRef{COR: first countable T1 topologies via their sequences}. The next two follow easily from the definition of $d_\text{new}$.
		\end{dgrsProof}
	\end{dgrs}
	
	
	\begin{prp}\label{PRP: any metric is topologically equiv to a bdd metric}
		Any metric is topologically equivalent to a bounded metric which preserves Cauchy sequences and total boundedness both ways.
	\end{prp}
	
	\begin{proof}
		Let $f\colon[0, +\infty)\to [0, 1)$ be a strictly increasing bijection which is subadditive, \ie, $f(x + y)\le f(x) + f(y)$,\footnote{
			This gives another reason why $f$ is continuous: $f(x + \delta) - f(x)\le f(\delta) \which < \epsilon$ if $\delta < f^{-1}(\epsilon)$.
		} for instance $x\mapsto x/(1 + x)$. Then $d'(x, y) := f(d(x, y))$ defines a bounded metric on $X$:
		\begin{prooflist}
			\item $d'(x, y) = 0\iff d(x, y) = f^{-1}(0)\which = 0$ (since \uline{$f$ a strictly increasing bijection}) $\iff x = y$.
			
			\item $d'$ is symmetric because $d$ is.
			
			\item $d'(x, y) + d'(y, z) = f(d(x, y)) + f(d(y, z))\ge f(d(x, y) + d(y, z))$ (since \uline{$f$ is subadditive}) $\ge f(d(x, z))$ (since \uline{$f$ is increasing}) $ = d'(x, z)$.
		\end{prooflist}
		
		Topological equivalence and preservation of total boundedness will follow if preservation of Cauchy sequences is established (\myRef{PRP: implications from unif equiv to topo equi} and \resp \myRef{LEM: totally bdd iff each seq has a Cauchy subseq}).
		
		That the Cauchy sequences are the same follows by continuity\footnote{
			Only continuity at $0$ is required.
		} of $f$ and $f^{-1}$ and the fact that $f(0) = 0 = f^{-1}(0)$ (due to the \uline{strict monotonicity and bijectivity of $f$}).
	\end{proof}
	
	\begin{rmk}
		Another choice of $f$ could have been $x\mapsto \min(x, 1)$, which although subadditive, is not bijective.
	\end{rmk}
	
	
	The following is an easy lemma:
	
	\begin{lem}
		Uniformly equivalent metrics preserve the following continuities: Usual, Cauchy-regular, uniform, Lipschitz.
	\end{lem}
	
	\begin{rmk}
		Note that just the topological equivalence (or even same Cauchy sequences) is not enough to preserve uniform continuity of even Lipschitz functions:\myMargin{
			What about Cauchy-regularity of Lipschitz?
		} COMPLETE THIS! \href{https://math.stackexchange.com/questions/1963784/uniform-continuity-preserved-with-equivalent-metrics}{SE link.}
	\end{rmk}
	



\subsection{Stronger forms of continuity}
	
	\begin{lem}
		Uniformly continuous functions preserve total boundedness.
	\end{lem}
	
	\begin{proof}
		Let $f\colon X\to Y$ be uniformly continuous with $X$ being totally bounded. Let $\epsilon > 0$. Due to \uline{uniform continuity}, take $\delta > 0$ such that $d(f(x), f(y)) < \epsilon$ whenever $d(x, y) < \delta$. Due to \uline{total boundedness}, let $B_\delta(x_1), \ldots, B_\delta(x_n)$ cover $X$. Now, $f(X) = \bigcup_i f(B_\delta(x_i))\subseteq \bigcup_i B_\epsilon(f(x_i))$.
	\end{proof}
	
	\begin{rmk}
		Boundedness needn't be preserved however: Consider $\id\colon(X, d_\text{discr})\to(X, d)$ where $d$ is any unbounded metric on $X$.
	\end{rmk}
	
	
	\begin{prp}\label{PRP: cont funcs on compact are unif cont}
		Continuous functions on compact sets are uniformly continuous.
	\end{prp}
	
	\begin{proof}
		Let $f\colon X\to Y$ be continuous with $X$ being compact. Let $\epsilon > 0$. For each $x\in X$, choose $\delta_x > 0$ such that $f(B_{\delta_x}(x))\subseteq B_\epsilon(f(x))$.\myMargin{No \AC needed!}
		Let $B_{\delta_{x_1}/2}(x_1), \ldots, B_{\delta_{x_n}/2}(x_n)$ cover $X$ (since \uline{$X$ is compact}). Now, any $x, y\in X$ lie in some $B_{\delta_{x_i}}(x_i)$ whenever $d(x, y) < \min(\delta_{x_1}, \ldots, \delta_{x_n})/2\wimplies d(f(x), f(y)) < 2\epsilon$.
	\end{proof}
	
	\begin{rmk}
		To see the necessity of compact domain, consider $x\mapsto 1/x$ on $\mathbb R\setminus\{0\}$.
	\end{rmk}
	
	
	\begin{prp}
		$f$ is uniformly continuous $\iff$ $d(f(x_n), f(y_n))\to 0$ whenever $d(x_n, y_n)\to 0$.
	\end{prp}
	
	\begin{proof}
		Consider $f\colon X\to Y$.
		
		``$\Rightarrow$'': Let $f$ be uniformly continuous. Let $d(x_n, y_n)\to 0$ in domain. Let $\epsilon > 0$. Take $\delta> 0$ such that $d(f(x), f(y)) < \epsilon$ whenever $d(x, y) < \delta$. Take $N$ such that for each $n\ge N$, we have $d(x_n, y_n) < \delta\wimplies d(f(x_n), f(y_n)) < \epsilon$.
		
		``$\Leftarrow$'': If $f$ is not not uniformly continuous, then we may take an $\epsilon > 0$ and for each $n$, choose\myMargin{
			\CC used!
		}
		$x_n, y_n\in E$ such that $d(x_n, y_n) < 1/n$ and yet $d(f(x_n), f(y_n)) \ge \epsilon$.
	\end{proof}
	

	\begin{prp}[Interaction of continuities]
		\leavevmode
		\begin{mylist}
			\item Lipschitz $\implies$ uniform continuity.
			
			\item Uniform continuity on every totally bounded subset of the domain $\iff$ Cauchy-regularity.
			
			\item Cauchy-regularity $\implies$ continuity.
		\end{mylist}
	\end{prp}
	
	\begin{proof}
		\begin{mylist}
			\item This is clear.
			
			
			\item ``$\Rightarrow$'': Since Cauchy sequences are totally bounded.
			
			``$\Leftarrow$'': Suppose $f$ is Cauchy-regular and yet not uniformly continuous on a totally bounded subset $E$ of the domain, so that we may take an $\epsilon > 0$ and for each $n$, choose\myMargin{
				\CC used!
			}
			$x_n, y_n\in E$ such that $d(x_n, y_n) < 1/n$ and yet $d(f(x_n), f(y_n)) \ge \epsilon$. \Wlogg, let $(x_n)$, $(y_n)$ be Cauchy (for $E$ is \uline{totally bounded}). Now, the sequence $x_1, y_1, x_2, y_2, \ldots$ is also Cauchy, and despite that, its $f$-image isn't.
			
			
			\item Let $f$ be Cauchy-regular and $x_i\to c$ in the domain. Then $x_1, c, x_2, c, \ldots$ is Cauchy $\wimplies$ $f(x_1), f(c), f(x_2), f(c), \ldots$ is Cauchy $\wimplies$ $f(x_i)\to f(c)$.\qedhere
		\end{mylist}
	\end{proof}
	
	\begin{rmk}
		\begin{mylist}
			\item $x\mapsto \sqrt x$ on $[0, 1]$ is uniformly continuous but not Lipschitz (observe at $0$).
			
			\item $x\mapsto x^2$ on $\mathbb R$ is Cauchy-regular and not uniformly continuous.
			
			\item $x\mapsto 1/x$ on $\mathbb R^+$ is continuous but not Cauchy-regular.
		\end{mylist}
	\end{rmk}
	
	
	\begin{thm}[Extension of Cauchy-regulars]\label{THM: extension of Chauchy-regs}
		A Cauchy-regular function from a dense subset to a complete codomain has a unique continuous extension to the whole of domain, which is further Cauchy-regular. Furthermore, this extension preserves uniform and Lipschitz continuities, and isometry-city.
	\end{thm}
	
	\begin{proof}
		Let $f\colon A\to Y$ be Cauchy-regular where $A$ is dense in $X$, and $Y$ complete. Let's first settle uniquness.\footnote{Which also directly follows from \myRef{PRP: cont func into Hausdorff uniquely by its vals on dense subset}.} Let $x\in X$. Then take a sequence\myMargin{
			\CC used; avoidable if $X$ separable.
		} $(a_i)\in A$ such that $a_i\to x$ (since \uline{$A$ dense in $X$}). If $\tilde f\colon X\to Y$ is a continuous extension of $f$, then we must have $f(a_i)\to \tilde f(x)$.
		
		Let's verify that this indeed gives a well-defined Cauchy-regular extension:
		\begin{prooflist}
			\item Well-defined:
			\begin{mylist}
				\item If $(a_i)$ is Cauchy in $A$, then by \uline{Cauchy-regularity}, it's $f$-image is also Cauchy, and thus \cgt due to \uline{completeness of $Y$}.
				
				\item Let $(a_i), (b_i)\in A$ converge to the same point in $X$. Then the interleaved sequence $a_0, b_0, a_1, b_1, \ldots$ is Cauchy. Due to \uline{Cauchy-regularity}, its $f$-image is also Cauchy $\wimplies$ $\lim_i f(a_i) = \lim_i f(b_i)$.
			\end{mylist}
			
			
			
			\item Extension:
			This is clear, since for $a\in A$, the constant sequence $(a, a, \ldots)$ converges to $a$ so that $\tilde f(a) = \lim_i f(a) = f(a)$.
			
			
			
			\item Cauchy-regularity:
			Let $(x^{(n)})\in X$ be Cauchy. We need to show that it's $\tilde f$-image is Cauchy as well. As before due to \uline{denseness of $A$}, choose Cauchy sequences\myMargin{
				\CC used twice; both avoidable if $X$ separable.
			} $( a^{(n)}_i )\in A$ such that $a^{(n)}_i\to x^{(n)}$ so that $\tilde f(x^{(n)}) = \lim_i b^{(n)}_i$. where $b^{(n)}_i := f(a^{(n)}_i)$. Note that
			\begin{mylist}
				\item the Cauchy-ness of $(x^{(n)})$ translates to $\lim_i d(a^{(m)}_i, a^{(n)}_i)\to 0$ as $m, n\to\infty$, and similarly,
				
				\item that of $(\tilde f(x^{(n)}))$ translates to $\lim_i d(b^{(m)}_i, b^{(n)}_i)\to 0$ as $m, n\to\infty$.
			\end{mylist}
			%
			Since the sequences are Cauchy, assume for all $n$'s \wlogg, that $d(a^{(n)}_j, a^{(n)}_i), d(b^{(n)}_j, b^{(n)}_i) < 1/i$ whenever $j\ge i$.
			
			Note that it suffices to get hold of a ``diagonal'' sequence $(b^{(n)}_{N_n})$ that is Cauchy with $N_n$'s increasing\footnote{
				Actually, what is required in the proof is just that $1/N_n\to 0$.
			} for then we'll have
			\begin{alignat*}{2}
				d(b^{(m)}_i, b^{(n)}_i)
				& \le d(b^{(m)}_i, b^{(m)}_{N_m}) + d(b^{(m)}_{N_m}, b^{(n)}_{N_n}) + d(b^{(n)}_{N_n}, b^{(n)}_i)\\
				& < 1/N_m + d(b^{(m)}_{N_m}, b^{(n)}_{N_n}) + 1/N_n & \text{(taking $i\ge N_m, N_n$)}
			\end{alignat*}
			so that we'll have $\lim_i d(b^{(m)}_i, b^{(n)}_i)$ being less than the \RHS which indeed goes to $0$ as $m, n\to \infty$ (since \uline{$(b^{(n)}_{N_n})$ Cauchy} and \uline{$N_n$'s increasing}).
			
			Since \uline{$f$ is Cauchy-regular}, it suffices to find a Cauchy $(a^{(n)}_{N_n})$. Choose $N_n$'s increasing, such that $d(a^{(n)}_i, a^{(n)}_n) < 1/n$ for each $i\ge n$. Now,
			\begin{alignat*}{2}
				d(a^{(m)}_{N_m}, a^{(n)}_{N_n})
				& \le d(a^{(m)}_{N_m}, a^{(m)}_{N_n}) + d(a^{(m)}_{N_m}, a^{(n)}_{N_n})\\
				& < 1/N_m + d(a^{(m)}_{N_m}, a^{(m)}_i) + d(a^{(m)}_i, a^{(n)}_i) & \text{(taking $n\ge m$)}\\
				& \phantom{{} < 1/N_m}\hspace{10pt} {} + d(a^{(n)}_i, a^{(n)}_{N_n})\\
				& < 2/N_m + d(a^{(m)}_i, a^{(n)}_i) + 1/N_n & \text{(taking $i\ge N_m, N_n$)}\\
				& \le 2/N_m + 1/N_n + \lim\nolimits_i d(a^{(m)}_i, a^{(n)}_i) & \text{(taking $i\to \infty$)}
			\end{alignat*}
			which indeed goes to $0$ as $m, n\to\infty$.
		\end{prooflist}
			
		Finally, we verify the preservations:
			\begin{prooflist}
				\item Preservation of uniform continuity:
				Let $f$ be uniformly continuous. We need to show that $\tilde f$ is also uniformly continuous. Let $\epsilon > 0$ and take $\delta > 0$ such that $d(f(a), f(b)) < \epsilon$ whenever $d(a, b) < \delta$ for $a, b\in A$. Now, let $x, y\in X$ with $d(x, y) < \delta$. Take\myMargin{
					Same comment on \CC.
				} $(a_i), (b_i)\in A$ converging to $x$, $y$ \resp. Now, $d(a_i, b_i) < \delta$ eventually (as $\lim_i d(a_i, b_i) = d(x, y) < \delta$) so that $d(f(a_i), f(b_i)) < \epsilon$ eventually $\wimplies$ $d(\tilde f(x), \tilde f(y)) \which= \lim_i d(f(a_i), f(b_i))\le \epsilon$.
				
				
				\item Preservation of Lipschitz continuity and isometry-city:
				Same technique as in the last point.
				\qedhere
			\end{prooflist}
	\end{proof}
	
	These immediately yields the following:
	
	\begin{cor}\label{COR: immediate conseqs of extension of Cauchy-regs}
		\begin{mylist}
			\item \itemHead{Characterizing Cauchy-regularity}
			A continuous function between metric spaces is Cauchy-regular $\iff$ it admits a continuous extension to their completions.
			
			\item \itemHead{Characterizing completions}\label{CORii: immediate conseqs of extension of Cauchy-regs}
			An isometry into a complete space is a completion $\iff$ its image is dense in the codomain.
			
			\item \itemHead{Completion of dense subsets}\label{CORiii: immediate conseqs of extension of Cauchy-regs}
			The completion of a dense subset is canonically obtained from that of the parent space.
			
			\item In a complete space, the closures of subsets are their completions.\footnote{
				Note how this is different from the familair fact that the closed sets of a complete space are precisely its complete subsets.
			}
		\end{mylist}
	\end{cor}
	
	\begin{proof}
		\begin{mylist}
			\item Obviously.
			
			\item ``$\Rightarrow$'' is the content of \myRef{PRP: a space is dense in its completion}. ``$\Leftarrow$'' follows since an isometry on a dense subset extends isometrically to the whole domain.
			
			\item Let $A$ be dense in $X$ and $\iota\colon X\to \hat X$ a completion of $X$. We show that the composition
			\begin{tikzcd}
				A\ar[r, hook] & X\ar[r, "\iota"] & \hat X
			\end{tikzcd}
			is a completion of $A$. Due to \myRef{CORii: immediate conseqs of extension of Cauchy-regs}, it suffices to have that $\iota(A)$ is dense in $\hat X$ $\wimpliedby$ $\iota(A)$ is dense in $\iota(X)$ (since \uline{$\iota(X)$ is already dense in $\hat X$}) $\wimpliedby$ \uline{$A$ is dense in $X$} (since $\iota$ is continuous; see \myRef{LEM: cont img of dense is dense}) which is indeed true.
			
			
			\item Follows from \myRef{CORiii: immediate conseqs of extension of Cauchy-regs}.
			\qedhere
		\end{mylist}
	\end{proof}
	
	
	
	
\subsection{Uniform convergence}

	Note that $d_\infty$ ``almost'' forms a metric on $X^E$ except that it can take infinite values. Thus, if $\mathscr F\subseteq X^E$ is such that $d_\infty(f, g) < +\infty$ for all $f, g\in \mathscr F$, then $d_\infty$ defines a metric on $\mathscr F$. It's easily seen that convergence under this metric coincides with uniform convergence.
	
	
	\begin{prp}
		Let $E$ be a topological space and $E_1\subseteq E$. Let $(f_n)$ be Cauchy in $X^{E_1}$.\footnote{
			That is, $d_\infty(f_m, f_n)\to 0$ as $m, n\to\infty$.
		} and $c\in E_1'$ with $f_n(x)\to L_n$ as $x\to c$. Then \tfh:
		\begin{mylist}
			\item $(L_n)$ is Cauchy.
			
			\item If $f_n\to f$ uniformly for $f\in X^{E_1}$ and $L_n\to L$ in $X$, then $f(x)\to L$ as $x\to c$.
		\end{mylist}
	\end{prp}
	
	\begin{proof}
		\begin{mylist}
			\item Let $\epsilon > 0$. Since \uline{$(f_n)$ Cauchy}, take $N$ such that $d_\infty(f_m, f_n) < \epsilon$ for all $m, n\ge N$. Now, let $m, n\ge N$. Since \uline{$f_m(x)\to L_m$ and $f_n(x)\to L_n$ as $x\to c$}, let $U$ be an \onbd of $c$ such that $d(f_m(x), L_m), d(f_n(x), L_n) < \epsilon$ for each $x\in E_1\cap U\setminus\{c\}$. Then
			\begin{alignat*}{2}
				d(L_m, L_n)
				& \le d(L_m, f_m(x)) + d(f_m(x), f_n(x)) \\
				& \phantom{{}\le d(L_m, f_m(x))} + d(f_n(x), L_n) & \text{(taking $x\in E_1$)}\\
				& < 3\epsilon\text, & \text{(taking $x\in U\setminus \{c\}$ as well)}
			\end{alignat*}
			where taking $x\in E_1\cap U\setminus\{c\}$ is allowed since \uline{$c\in E_1'$}.
			
			
			\item Let $\epsilon > 0$. Since \uline{$f_n\to f$} and \uline{$L_n\to L$}, take $N$ such that $d_\infty(f_N, f), d(L_N, L) < \epsilon$. Since \uline{$f_N(x)\to L_N$ as $x\to c$}, take an \onbd $U$ of $c$ such that $f_N(U\setminus\{c\})\subseteq B_\epsilon(L_N)$. Now, for $x\in E_1\cap U\setminus\{c\}$, we have
			\begin{alignat*}{2}
				d(f(x), L)
				& \le d(f(x), f_N(x)) + d(f_N(x), L_N) + d(L_N, L)\\
				& < 3\epsilon\text. \tag*{\qedhere}
			\end{alignat*}
		\end{mylist}
	\end{proof}
	
	\begin{cor}
		Uniform convergence preserves continuity.
	\end{cor}
	
	This is \ofc not true of just pointwise convergence.
	
	
	% - COMPLETENESS OF F IF X IS COMPLETE !!!
	% - CLEANER PROOF OF COMPLETION OF METRIC SPACES --- motivated from NLS's
	%		duals
	
	
	


\section{Baire's Category Theorem}

	\begin{prp}[Cantor's intersection]
		In a complete metric space, the intersection of a decreasing sequence of closed subsets with diameters going to zero, is a singleton.
	\end{prp}
	
	\begin{proof}
		Let $F_i$'s be the closed sets under consideration. That there's at most one point in the intersection is clear since \uline{$\delta(F_i)\to 0$}. Now, choose\myMargin{\CC used.}
		$x_i\in F_i$, which form a Cauchy sequence since \uline{$\delta(F_i)\to 0$}. Since the space is \uline{complete}, let $x_i\to x$, and since \uline{each $F_i$ is closed}, $x$ lies in the intersection.
	\end{proof}
	
	\begin{rmk}
		The necessity of each hypothesis is easy to see.
	\end{rmk}
	
	\begin{dgrs}
		The diameter of the intersection of a decreasing sequence of subsets needn't be the corresponding limit of diameters even if the sets are closed and bounded. For instance, consider an infinite dimensional \NLS containing orthonormal vectors $e_1, e_2, \ldots$. Take $F_i := \{e_i, e_{i + 1}, \ldots\}$. Then each $\delta(F_i) = \sqrt 2$, and still the intersection is empty. However, there is one case where we \emph{can} say something:
		
		\begin{prp}
			Let $F_1\supseteq F_2\supseteq\cdots$ be closed subsets of a metric space with $F_1$ being compact. Then $\delta(\bigcap_i F_i) = \lim_i\delta(F_i)$.
		\end{prp}
		
		% COMPACTNESS => SEQUENTIAL COMPACTNESS
		\begin{dgrsProof}
			``$\le$'' is clear. For ``$\ge$'', let $\epsilon> 0$ and choose $x_i, y_i\in F_i$ such that $d(x_i, y_i) > \delta(F_i) - \epsilon$ (note that each $\delta(F_i) < +\infty$). Now, since \uline{$F_1$ is compact}, let $x_{n_i}\to x$ and $y_{n_i}\to y$ in $F_1$. Since \uline{$F_i$'s are closed}, $x$, $y$ lie in the intersection so that $\delta(\bigcap_i F_i)\ge d(x, y)\ge \lim_i\delta(F_i) - \epsilon$.
		\end{dgrsProof}
	\end{dgrs}
	
	\begin{thm}[Baire's category]\label{THM: BCT}
		In a complete metric space, complements of meager sets are dense.
	\end{thm}
	
	\begin{proof}
		Let $A_1, A_2, \ldots$ be nowhere dense. We show that $X\setminus \bigcup_i A_i$ is dense. Pick a nonempty open $U$. Since \uline{$A_1$ is nowhere dense}, choose $x_1\in U$ and $r_1 > 0$ such that $B_{r_1}(x_1)\subseteq U$ and $B_{r_1}(x_1)\cap A_1 = \emptyset$. Having chosen\myMargin{\DC used.}
		$x_i$, $r_i$, choose $x_{i + 1}\in B_{r_i}(x_i)$ such that
		\begin{assmplist}
			\item $B_{r_{i + 1}}(x_{i + 1})\subseteq B_{r_i}(x_i)$,
			\item $r_{i + 1}\le r_i/2$, and
			\item $B_{r_{i + 1}}(x_{i + 1})\cap A_{i + 1} = \emptyset$.
		\end{assmplist}
		This is possible since \uline{$A_i$ is nowhere dense}.
		Thus, $\clos{B_{r_1}(x_1)}\supseteq\clos{B_{r_2}(x_2)}\supseteq\cdots$ with $\delta(\clos{B_{r_i}(x_i)})\le\delta(D_{2r_i}(x_i)) \which = 2r_i\which\to 0$ since $r_i\le r_1/2^{i - 1}$. By Cantor (since \uline{$X$ is complete}), let $x\in\bigcap_i\clos{B_{r_i}(x_i)}$. Then $x\notin \bigcup_i A_i$ since each $B_{r_i}(x_i)\cap A_i = \emptyset$. Finally, $x\in \clos{B_{r_1}(x_1)}$ and \wlogg, we could've chosen $x_1$, $r_1$ such that $\clos{B_{r_1}(x_1)}\subseteq U$.
	\end{proof}
	
	\begin{rmk}
		Necessity of completeness is demonstrated by considering any countable metric space in which singletons are not open, for instance $\mathbb Q$.
	\end{rmk}
	
	\begin{cor}
		If countably many closed subsets of a nonempty complete metric space unite to the whole space, then one of them has a nonempty interior.
	\end{cor}
	
	
	
	
	
	
	
	
	
	
	
	