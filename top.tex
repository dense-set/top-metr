\chapter{Topology}

\begin{conv}
	Unless stated otherwise,
	\begin{assmplist}
		\item $X$, $Y$ will be topological spaces.
		
		\item Subsets of topological spaces will be considered under subspace topology.
		
		\item Product of topological spaces will be considered under product the topology.
		
		\item A totally ordered set will be considered under the order topology. (See \myRef{SEC: order topo}.)
		
		\item Monotonic functions will be assumed to be between totally ordered spaces.
	\end{assmplist}
\end{conv}



% needs more stuff
\section{Subspaces and Bases}

	\begin{lem}
		$\mathscr B$ is a base $\iff$ the arbitrary unions in $\mathscr B$ form a topology.\myMargin{``$\Rightarrow$'' requires \AC.}
	\end{lem}
	
	\begin{lem}\label{LEM: subspaces and bases}
		\leavevmode
		\begin{mylist}
			\item ``Being a subspace of'' is transitive.
			
			\item\label{LEMii: subspaces and bases} (Sub)base of a subspace can be obtained from that of the parent space.
		\end{mylist}
	\end{lem}
	



% add more stuff
\section{Limits and Continuity}

	Let $E\subseteq X$ and $f\colon E\to Y$. Then for any $c\in X$ and $L\in Y$, we write \defn{$f(x)\to L$ as $x\to c$ in $X$} iff for every \onbd $V$ of $L$ in $Y$, there exists an \onbd $U$ of $c$ in $X$ such that $f(E\cap U\setminus\{c\})\subseteq V$. Note that ``in $X$'' is crucial and can't be dropped: Consider $\id$ on $(0, 1)$. Take $X_1$ to be the disjoint union topology of $(0, 1)$ and $\{1\}$ and $X_2 := (0, 1]$ under the subspace topology. Then $1$ is isolated in $X_1$ and thus $f(x)\to L$ for each $L\in (0, 1)$ as $x\to 1$ in $X_1$ (see \ref{LEMi: lims at isolated and lim pts} of \myRef{LEM: lims at isolated and lim pts}). \Otoh, $f(x)\to 1$ only as $x\to 1$ in $X_2$ since $1$ is a limit point of $(0, 1)$ in $X_2$ (see \ref{LEMii: lims at isolated and lim pts} of \myRef{LEM: lims at isolated and lim pts}).
	
	Intuitively, what this shows is that $c$'s topological relation with $E$ is not determined by $E$ alone if $c\notin E$, which is plausible. However, if $c\in E$, then specification of ambient space is redundant (see \ref{LEMiii: lims at isolated and lim pts} of \myRef{LEM: lims at isolated and lim pts}).
	
	Nevertheless, we omit ``in $X$'' if the ambient space is clear from the context, or if the point at which the limit is being evaluated lies in the domain.
	
	\begin{lem}\label{LEM: lims at isolated and lim pts}
		\begin{mylist}
			\item\label{LEMi: lims at isolated and lim pts} All the codomain values are the limits of a function at an isolated point of the domain.
			
			\item\label{LEMii: lims at isolated and lim pts} For Hausdorff codomains, there is at most one limit at a limit point of the domain.
			
			\item\label{LEMiii: lims at isolated and lim pts} Limits of a function at a point in the domain are independent of the ambient space.
		\end{mylist}
	\end{lem}
	
	\begin{proof}
		Let $f\colon E\to Y$ where $E\subseteq X$. Let $c\in X$.
		\begin{mylist}
			\item If $c$ is isolated in $E$, then there exists an \onbd $U$ of $c$ in $X$ such that $E\cap U = \{c\}$ so that for any open set $V$ in $Y$, we have $f(E\cap U\setminus\{c\}) = \emptyset\subseteq V$.
			
			\item Let $c$ be a limit point of $E$ and suppose $f(x)\to L_1, L_2$ as $x\to c$ for distinct $L_1$, $L_2$. Let $V_i$'s be opens separating $L_i$'s for $i = 1, 2$ (since \uline{$Y$ Hausdorff}). Now, take \onbd{s} $U_i$'s of $c$ in $X$ such that $f(E\cap U_i\setminus\{c\})\subseteq V_i$. Now, $E\cap U_1\cap U_2\setminus\{c\}\ne\emptyset$ set \uline{$c$ is a limit point of $E$} which violates disjointness of $V_i$'s.
			
			\item Since if $c\in E$, then for any $X_1, X_2\supseteq E$, if $U_1$ is an \onbd of $c$ in $X_i$, then $E\cap U_1$, being an \onbd of $c$ in $E$, is equal to $E\cap U_2$ for some \onbd $U_2$ of $c$ in $X_2$.\qedhere
		\end{mylist}
	\end{proof}
	
	\begin{rmk}
		This allows to use $\lim_{x\to c} f(x)$ notation in case $f$ has a Hausdorff codomain and $c$ is a limit point of the domain in the ambient space.
	\end{rmk}
	
	
	\begin{lem}[Limits and subspaces]\label{LEM: lims and subspaces}
		Assume the following:
		\begin{assmplist}
			\item $f\colon E\to Y$ where $E\subseteq X$.
			\item $c\in A\subseteq X$ and $L\in Y$.
			\item $f|\colon E\cap A\to Y$.
		\end{assmplist}
		Then \tfh:
		\begin{mylist}
			\item $f(x)\to L$ as $x\to c$ in $X$ $\implies$ $f|(x)\to L$ as $x\to c$ in $A$.
			
			\item\label{LEMii: lims and subspaces} The converse holds if $A$ is open.
		\end{mylist}
	\end{lem}
	
	\begin{proof}
		\begin{mylist}
			\item Let $V$ be an \onbd of $L$. Take an \onbd $U$ of $c$ in $X$ such that $f(U\cap E\setminus\{c\})\subseteq V$. Then $U\cap A$ is an \onbd of $c$ in $A$ with $f|(U\cap E\cap A\setminus\{c\})\subseteq V$.
			
			\item Let $A$ be open. Let $V$ be an \onbd of $L$. Then take an \onbd $U$ of $c$ in $A$ such that $f|(U\cap E\setminus\{c\})\subseteq V$. Now, just note that $U$ is open in $X$ as well since \uline{$A$ is open}.
			\qedhere
		\end{mylist}
	\end{proof}
	
	\begin{rmk}
		Necessity of $A$ being open in \ref{LEMii: lims and subspaces} is clear by taking $Y$ to be Hausdorff containing at least two points, $A$ to be a singleton comprising of a limit point of $E$ and taking $f$ appropriately (constant function works).
	\end{rmk}
	
	
	\begin{prp}[Pointwise pasting]
		Assume the following:
		\begin{assmplist}
			\item $f\colon E\to Y$ where $E\subseteq X$.
			\item $X$ is the union of finitely many closed $F_i\subseteq X$.
			\item $f|_i : E\cap F_i\to Y$.
			\item $c\in X$ and $L\in X$.
			\item If $c\in F_i$, then $f_i(x)\to L$ as $x\to c$ in $F_i$.
		\end{assmplist}
		Then $f(x)\to L$ as $x\to c$ in $X$.
	\end{prp}
	
	\begin{proof}
		Let $V$ be an \onbd of $L$. Say $c\in F_i$'s and $c\notin F_j$'s. Thus, take \onbd{s} $U_i\cap F_i$ of $c$ in $F_i$ ($U_i$ open in $X_i$) such that $f|_i(U_i\cap E\cap F_i\setminus\{c\})\subseteq V$. Set $U := \bigl( \bigcap_i U_i \bigr)\cap \bigl(X\setminus \bigcup_j F_j\bigr)$\footnote{
			The second set in the union is motivated from the need to have $\bigcap_j X\setminus F_j$.
		} which is open since $F_j$'s are \uline{closed} and $i$ and $j$'s are \uline{finitely many}. Thus, $U$ is an \onbd of $c$ in $X$ with $f(U\cap E\setminus\{c\}) 
		= \bigl( \bigcup_i f(U\cap E\cap F_i\setminus\{c\}) \bigr)
			\cup \bigl( \bigcup_j f(U\cap E\cap F_j\setminus\{c\}) \bigr)
		\subseteq V\cup \emptyset = V$.
	\end{proof}
	
	\begin{rmk}
		\begin{mylist}
			\item Necessity of finitely many closed sets: Consider $f\colon \mathbb R^2\to\mathbb R$ given by $f(x, y) := x^2y/(x^4 + y^2)$ for $(x, y)\ne (0, 0)$ and $f(0, 0) := 0$. Along all the straight lines through origin (which are closed), $f(x)\to 0$ and yet $f$ has no limit at $0$.\footnote{
				Consider evaluating along $y = mx^2$.
			}
			
			A much simpler example is to consider an infinite $X$ having at least one limit point, the singleton containing which is closed, and taking the codomain to be a Hausdorff space with at least two points.
			
			%====!!!!====%
			\item Necessity of all closed:
		\end{mylist}
	\end{rmk}





\section{Product Topology}

	From \ref{LEMii: subspaces and bases} of \myRef{LEM: subspaces and bases}, we immediately conclude:

	\begin{lem}
		Taking products and subspaces are compatible.
	\end{lem}
	
	\begin{rmk}
		This holds for box topology as well.
	\end{rmk}
	
	\begin{lem}
		Closure of a product is the product of closures.
	\end{lem}
	
	\begin{proof}
		Let $A_i\subseteq X_i$. We show $\clos{\prod_i A_i} = \prod_i\clos{A_i}$.
		
		``$\subseteq$'': Suffice to show that $\prod_i F_i$ is closed for $F_i$'s closed in $X_i$'s. Let $(x_i)\notin\prod_i F_i$, say $x_{i_0}\notin F_{i_0}$. Then take an \onbd $U_{i_0}$ of $x_{i_0}$ disjoint from $F_{i_0}$. Now, $\pi_{i_0}^{-1}(U_{i_0})$ is an \onbd of $(x_i)$ that is disjoint from $\prod_i F_i$.
		
		``$\supseteq$'': Let $U := \bigcap_{j\in J}\pi_j^{-1}(U_j)$ be an \onbd of $(x_i)\in\RHS$, where $J$ is finite and each $U_j$ is open. Then each $U_j$ is an \onbd of $x_j$ and hence intersects $A_j$. Thus $U$ intersects $\prod_i A_i$.\myMargin{No choice \reqd.}
	\end{proof}
	
	\begin{rmk}
		The same holds for box topology as well; however \AC will be required for ``$\supseteq$''.
	\end{rmk}
	
	\begin{cor}
		Product of dense sets is dense in the product.
	\end{cor}
	
	\begin{rmk}
		This holds for box topology as well.
	\end{rmk}
	
	
	\begin{prp}[Convergence]\label{PRP: convergence in prod topo}
		Convergence in product topology is equivalent to convergence of each component sequence in the respective factor space.
	\end{prp}
	
	\begin{proof}
		``$\Rightarrow$'': Since projections are continuous.
		
		``$\Leftarrow$'': Let $X$ be the product of $X_i$'s and $x, (x^{(n)})\in X$ be such that $x^{(n)}_i\to x_i$ for each $i$. We show that $x^{(n)}\to x$. Let $U := \bigcap_j \pi_j^{-1}(U_j)$ be a basic \onbd of $x$ where $j$'s are finitely many. Thus, $(x^{(n)}_j)_n$ eventually lies in $U_j$ for each $j$. Since there are finitely many $j$'s, we conclude that $(x^{(n)})$ eventually lies in $U$.
	\end{proof}
	
	\begin{rmk}
		This does not hold for box topology: Consider the $\mathbb N$-fold product of discrete $\{0, 1\}$ endowed with box topology and consider the sequence whose $n$-th term is given by $(\underbrace{1, \ldots, 1}_\text{$n$ times}, 0, 0, \ldots)$.
	\end{rmk}
	
	


\section{Order Topology}\label{SEC: order topo}

	If $X$ is totally ordered, then the \defn{order topology} on it is generated by these sets:
	\begin{rmklist}
		\item $(a, b)$;
		\item $[\min X, b)$ if $X$ has a minimum \elt; and,
		\item $(a, \max X]$ if $X$ has a maximum \elt.
	\end{rmklist}
	It's easily seen that these form a base that is closed under pairwise intersections.

	\begin{lem}[Immediate properties]\label{LEM: triv things on order topo}
		\leavevmode
		\begin{mylist}
			\item Open rays are open in order topology.
			\item Order topology is Hausdorff.
			\item\label{LEMiii: triv things on order topo} Topology induced from inherited order is coarser than the subspace topology. However, on a convex subset, then the topologies coincide.
			
			\item\label{LEMiv: triv things on order topo} The order topology coincides with the discrete topology $\iff$ each non-least \elt has an immediate predecessor and each non-greatest \elt has an immediate successor.
		\end{mylist}
	\end{lem}
	
	\begin{proof}
		\begin{mylist}
			\item Let's show for right-rays. In case there's a largest \elt, then it's clear. If not, then $(a, +\infty) = \bigcup_y (a, y)$, which is open.
			
			\item Let $x < y$. If there's a $z$ between them, then $(-\infty, z)$ and $(z, +\infty)$ separate them. Otherwise, $(-\infty, y)$ and $(x, +\infty)$ do.
			
			\item First statement is obvious. Second is tedious, but easy.
			
			\item ``$\Leftarrow$'' is easy. For ``$\Rightarrow$'', just note that a base for discrete topology must contain all the singletons.
			\qedhere
		\end{mylist}
	\end{proof}
	
	\begin{rmk}
		To see strict inclusion in \ref{LEMiii: triv things on order topo}, consider $\{-1\}\cup(0, 1]\subseteq \mathbb R$.
	\end{rmk}
	
	Let $f\colon X\to Y$ where $X$ is totally ordered. Then for $c\in X$ and $L\in Y$, we write \defn{$f(x)\to L$ as $x\to c^-$} iff $f|(x)\to L$ as $x\to c$ where $f|\colon (-\infty, c]\to Y$. Note:
	\begin{mylist}
		\item The domain $(-\infty, c]$ can be taken under either the subspace topology or the topology due to the induced order, which coincide. (See \ref{LEMiii: triv things on order topo} of \myRef{LEM: triv things on order topo}.)
		
		\item One of the reasons to include $c$ in the domain of $f|$ is that we don't have to bother about the ambient space. (See \ref{LEMiii: lims at isolated and lim pts} of \myRef{LEM: lims at isolated and lim pts}.)
	\end{mylist}
	
	\Lly, we define \defn{$f(x)\to L$ as $x\to c^+$}.
	
	
	\begin{lem}\label{LEM: limits via the left and right hand limits}
		Let $X$ be totally ordered and $f\colon X\to Y$. Let $c\in X$ and $L\in Y$. Then \tfae:
		\begin{mylist}
			\item\label{LEMi: left and right hand limits} $f(x)\to L$ as $x\to c$.
			
			\item\label{LEMii: left and right hand limits} $f(x)\to L$ as $x\to c^+$ or as $x\to c^-$.
		\end{mylist}
	\end{lem}
	
	\begin{proof}
		\ref{LEMi: left and right hand limits} $\Rightarrow$ \ref{LEMii: left and right hand limits}:
		Let 
	\end{proof}
	
	\begin{rmk}
		This lemma is like the pasting lemma, but at a point.
	\end{rmk}
	
	
	\begin{prp}\label{PRP: one-sided lims of monotonics}
		Let $X$, $Y$ be totally ordered spaces and $f\colon X\to Y$ be monotonic. Then for any $c\in X$, each of \tfh{s} whenever the \RHS is defined:
		\begin{alignat*}{2}
			f(x)& \to \sup_{x < c}f(x) &&\text{ as } x\to c^-\\
			f(x)& \to \inf_{x > c}f(x) &&\text{ as } x\to c^+
		\end{alignat*}
	\end{prp}
	
	\begin{proof}
		Let $f$ be increasing. The proof for decreasing $f$ would be similar. We show only the first statement, second's proof being similar. Let $\alpha$ be the \lub\ of $\{f(x) : x < c\}$. We need to show that $f|(x)\to \alpha$ as $x\to c$ where $f|\colon (-\infty, c]\to Y$. Let $J$ be a basic \onbd of $\alpha$. We have two cases:
		\begin{mylist}
			\item $\alpha$ is the least \elt of $Y$: Then $f((-\infty, c)) \subseteq \{\alpha\}\subseteq J$.
			
			\item $\alpha$ is not the least \elt: Then \wlogg, let $J$'s left-end be open at $y < \alpha\which = \sup_{x <c} f(x)$. Thus, take an $x < c$ such that $f(x) > y$. Now, due to \uline{monotonicity}, $f((x, c))\subseteq (y, \alpha]\subseteq J$.\qedhere
		\end{mylist}
	\end{proof}
	
	\begin{cor}\label{COR: one-sided lims of monotonics in a complete codom}
		Monotonics taking values in a complete codomain admit one-sided limits.
	\end{cor}
	
	\begin{proof}
		Let $f\colon X\to Y$ be increasing and $c\in X$. We find an $L\in Y$ such that $f(x)\to L$ as $x\to c^-$. Simply observe that $\{f(x) : x < c\}$ is bounded above by $f(c)$. We have two cases:
		\begin{mylist}
			\item $c$ is not the least \elt of $X$: Then $\{f(x) : x < c\}$ is nonempty as well so that we may take $L$ to be its \lub\ (which exists since \uline{$Y$ is complete}).
			
			\item $c$ is the least \elt of $X$: Then the domain of $f|\colon(-\infty, c]\to Y$ is the singleton $\{c\}$ and thus we may take $L$ to be any point in $Y$.
		\end{mylist}
		The proofs for other cases are similar.
	\end{proof}
	
	\begin{rmk}
		To see the necessity of well-definedness of \RHS in \myRef{PRP: one-sided lims of monotonics} and that of completeness in \myRef{COR: one-sided lims of monotonics in a complete codom}, consider $f\colon \mathbb R\to\mathbb R\setminus\{0\}$ given by
		\[
		f(x) := 
		\begin{cases}
			x, & x < 0\\
			x + 1, & x \ge 0
		\end{cases}\text.
		\]
	\end{rmk}
	
	
	\begin{prp}[Continuity and monotonicity]
		\leavevmode
		\begin{mylist}
			\item Strict monotonic surjections are homeomorphisms.
			
			\item A monotonic surjection is continuous provided the codomain's order is dense.
		\end{mylist}
	\end{prp}
	
	\begin{proof}
		\begin{mylist}
			\item Since inverses of strictly monotonic bijections are strict monotones as well, it suffices to show that strictly monotonic surjections are open which is easily shown.
			
			\item Let $X$, $Y$ be totally ordered and $f\colon X\to Y$ be an increasing surjection. The proof for decreasing $f$ would be similar. Let $J$ be a basic open set of $Y$. We show that $f^{-1}(J)$ is open. Start with an $x\in f^{-1}(J)$, \ie, $f(x)\in J$. The following cases arise:
			\begin{mylist}
				\item $f(x)$ is least in $Y$: Then \wlogg, take $J = [f(x), y)$ for some $y > f(x)$. Now, take $f(x) < v < y$ due to \uline{denseness}. Due to \uline{surjectivity}, take $b\in X$ such that $f(b) = v$. Since $f$ is \uline{increasing}, we have $x < b$ and $f([x, b))\subseteq [f(x), f(b)] = [f(x), v]\subseteq [f(x), y) = J$. Now, take an \onbd $I$ of such that $I\cap [x, +\infty)\subseteq [x, b)$. Then, since $f((-\infty, x]) = \{f(x)\}$ (as \uline{$f(x)$ is the least \elt}) $f(I)\subseteq J\wimplies I\subseteq f^{-1}(J)$.
				
				\item $f(x)$ is the greatest in $Y$: \Lly as above.
				
				\item $f(x)$ is neither: Then \wlogg, take $J = (y_1, y_2)$ for $y_1 < f(x) < y_2$. As above, due to \uline{denseness} and \uline{surjectivity}, take $a, b\in X$ such that $y_1 < f(a) < f(x) < f(b) < y_2$. Again, since $f$ is \uline{increasing}, $a < x < b$ and $f((a, b))\subseteq [f(a), f(b)]\subseteq (y_1, y_2) = J$.
				\qedhere
			\end{mylist}
		\end{mylist}
	\end{proof}
	
	\begin{rmk}
		\begin{mylist}
			\item Necessity of surjectivity: $f\colon \mathbb{R\to R}$ given by $f(x) := x$ for $x < 0$ and $f(x) := x + 1$ for $x\ge 0$.
			
			\item Necessity of denseness of codomain: Consider the sign function $\mathbb R\to \{-1, 0, 1\}$. (Note that the codomain is discrete due to \ref{LEMiv: triv things on order topo} of \myRef{LEM: triv things on order topo}.)
		\end{mylist}
	\end{rmk}
	



\section{Denseness}

	\begin{lem}\label{LEM: denseness is transitive}
		``Being dense'' is transitive.
	\end{lem}
	
	\begin{proof}
		Let $A\subseteq B\subseteq X$ with $A$ dense in $B$ and $B$ dense in $X$. Let $U$ be a nonempty open in $X$. Then \uline{$B$ being dense}, intersects $U$ so that $U\cap B$ is a nonempty open in $B$ and thus is intersected by the \uline{dense $A$} $\wimplies$ $A$ intersects $U$.
	\end{proof}
	
	
	\begin{lem}\label{LEM: prd of dense is dense}
		Product of dense sets is dense in the product.
	\end{lem}
	
	\begin{proof}
		Let $A_i$ be dense in $X_i$. Let $U_i$'s be nonempty open in $X_i$'s. Then $\prod_i A_i\cap \prod_i U_i = \prod_i A_i\cap U_i$, which is nonempty since each $A_i\cap U_i$ is nonempty.\myMargin{
			\AC used.
		}
	\end{proof}
	
	\begin{rmk}
		This is also true of the box topology.
	\end{rmk}
	
	
	
	\begin{lem}\label{LEM: dense on a subset}
		Let $A, B\subseteq X$. Then \tfh:
		\begin{mylist}
			\item $B\cap A$ is dense in $B$ $\implies$ $B\subseteq \clos A$.
			
			\item\label{LEMii: dense on a subset} The converse holds if $B$ is open.
		\end{mylist}
	\end{lem}
	
	\begin{proof}
		\begin{mylist}
			\item We have $B = \cl_B(B\cap A)\subseteq B\cap \clos{B\cap A}\subseteq\clos{B\cap A}\subseteq\clos A$.
			
			\item We need to show that $\cl_B(B\cap A) = B$. Indeed, if $F$ is any closed such that $B\cap A\subseteq B\cap F$, then $B\subseteq F$ (otherwise, take $x\in B\setminus F\wimplies x\in B\setminus A\wimplies x\in B\setminus \clos A$ for \uline{$B$ is open}, contradicting $B\subseteq \clos A$).\qedhere
		\end{mylist}
	\end{proof}
	
	\begin{rmk}
		To see the necessity of openness of $B$ in \ref{LEMii: dense on a subset}, consider $A = \{1, 1/2, \ldots\}$ and $B = \{0\}$.
	\end{rmk}
	
	
	
\subsection{Nowhere dense sets}
	
	\myRef{LEM: dense on a subset} gives insight as to why nowhere dense sets are called so---they are dense on no nonempty \emph{open} set. On the other hand, dense sets are dense on the whole space.
	
	\begin{lem}
		Let $U$ be open in $X$ and $A\subseteq X$. Then \tfae:
		\begin{mylist}
			\item $U\subseteq \clos A$.
			
			\item Every nonempty open subset contained in $U$ intersects $\clos A$.
			
			\item Every nonempty open subset contained in $U$ intersects $A$.
		\end{mylist}
	\end{lem}
	
	\begin{cor}\label{COR: characterizing nowhere dense}
		\Tfae for a subset $A$ of $X$:
		\begin{mylist}
			\item $X\setminus \overline A$ is dense.
			
			\item $A$ is nowhere dense.
			
			\item Each nonempty open set contains a nonemtpy open subset disjoint from $\overline A$.
			
			\item Each nonempty open set contains a nonemtpy open subset disjoint from $A$.
		\end{mylist}
	\end{cor}
	
	Subsets of a topological space that are countable unions of nowhere dense sets are called \defn{first category} or \defn{meagre} sets. Others are called \defn{second category} sets.
	
	\begin{rmk}
		In $\mathbb R$:\footnote{Nonmeagre-ness can be concluded by Baire's category theorem (\myRef{THM: BCT}).}
		\begin{center}
			\begin{tabular}{r|cc}
				& meagre & nonmeagre\\
				\hline
				dense & $\mathbb Q$ & $\mathbb R$\\
				nondense & $\emptyset$ & $[0, 1]$
			\end{tabular}
		\end{center}
	\end{rmk}
	
	\begin{lem}
		If $F_1, F_2, \ldots$ are closed in $X$ with $X\setminus\bigcup_i F_i$ dense, then each $F_i$ is nowhere dense.
	\end{lem}
	
	\begin{rmk}
		Baire's category theorem (\myRef{THM: BCT}) gives a converse to above, stating that complements of meagre sets are dense in a complete metric space.
	\end{rmk}
	
	\begin{prp}\label{PRP: versions of BCT}
		In a topological space, \tfae:
		\begin{mylist}
			\item\label{PRPi: versions of BCT} Complements of meagre sets are dense.
			\item\label{PRPii: versions of BCT} Countable intersections of open dense sets are dense.
		\end{mylist}
	\end{prp}
	
	\begin{proof}
		`` $\Rightarrow$ '': Let $U_1, U_2, \cdots$ be open dense. Now, $\bigcap_i U_i \which= X\setminus\bigcup_i(X\setminus U_i)$ is dense if each $X\setminus U_i$ is nowhere dense $\wimpliedby$ $X\setminus(\overline{X\setminus U_i})\which= U_i$ (since \uline{$U_i$ open}) is \uline{dense}, which is true.
		
		`` $\Leftarrow$ '': Let $A_1, A_2, \ldots$ be nowhere dense. Then each $X\setminus\overline A_i$ is dense $\wimplies$ $\bigcap_i(X\setminus \overline A_i)\which= X\setminus\bigcup_i\overline A_i$ is dense $\wimplies$ $X\setminus\bigcup_i A_i$ is dense as well, being a larger set.
	\end{proof}
	
	
	
	
\section{Connectedness}

	\begin{lem}[Characterizing disconnectedness]
		$E\subseteq X$ is disconnected $\iff$ $E$ can be written as a union of two nonempty subsets $A$, $B$ of $X$ such that $\overline A\cap B = \emptyset = A\cap\overline B$.
	\end{lem}
	
	\begin{proof}
		``$\Rightarrow$'': Take $U$, $V$ open in $X$ such that $E\cap U$, $E\cap V$ are nonempty, $E\subseteq U\cup V$, and $E\cap U\cap V = \emptyset$. Now put $A := E\cap U$ and $B := E\cap V$. Then $\overline A\cap B\subseteq \overline{E\cap U}\cap V = \emptyset$.
		
		``$\Leftarrow$'': Take $U := X\setminus \overline A$ and $V := X\setminus \overline B$. Then $B\subseteq U$ and $A\subseteq V$ so that both are nonempty and $E\subseteq U\cup V$. Also, $E\cap U\cap V = E\setminus(\overline A\cup\overline B) = \emptyset$.
	\end{proof}
	
	\begin{prp}[Linear continua are connected]\label{PRP: linear continua are connected}
		Then the connected subsets of a densely and completely totally ordered space\myMargin{
			Can we improve to partial orders?
		} are precisely its convex subsets.\footnote{Recall that a convex subset of an ordered set is any set $I$ such that $[x, y]\subseteq I$ whenever $x, y\in I$ with $x\le y$.}
	\end{prp}
	
	\begin{proof}
		Let $X$'s topology come from a dense and complete total order. Suppose $I\subseteq X$ is convex, and yet separated by opens $U$, $V$. Take $a\in U\cap I$ and $b\in V\cap I$. \Wlogg, assume $a < b$ (\uline{the order is total}) so that $[a, b]\subseteq I$ (since \uline{$I$ is convex}). Note that $U$, $V$ also form a separation of $[a, b]$. Since $U\cap [a, b]$ is nonempty and bounded, let $c$ be its \uline{\lub} Clearly, $c\in[a, b]$ so that there are two cases:
		\begin{mylist}
			\item[$c\in U$:] Take a basic interval $J\subseteq U$ containing $c$. Note that $c < b$ (since $b\in V$) so that $J\supseteq [c, d)$ for some $d > c$. Hence, $U\cap [a, b]\which\supseteq J\cap [a, b]\supseteq[c, d)\cap[c, b] \which= [c, d_1)$ where $d_1 :=  \min(d, b) > c$. Now take \uline{$e$ between $c$ and $d_1$}. Then $e\in U\cap[a, b]$ despite $e > c$.\myMargin{Add a diagram!}
			
			\item[$c\in V$:] Again take a basic interval $J\subseteq V$ containing $c$. This time, $c > a$ (as $a\in U$) so that $J\supseteq (d, c]$ for some $d < c$. Thus, $V\cap[a, b]\which\supseteq J\cap[a, b]\supseteq (d, c]\cap [a, c]\which=(d_1, c]$ where $d_1 := \max(d, a) < c$. Now, take \uline{$e$ between $d_1$ and $c$}. Then $e$ is an \ub for $U\cap[a, b]$ greater than $c$:
			\begin{subproof}
				If $x\in U\cap[a, b]$ is greater than $e$, then $x\in (e, c]\subseteq(d_1, c]\subseteq V$.
			\end{subproof}
		\end{mylist}
		
		Conversely, if $I$ is not convex, then take $x < y < z$ such that $z, z\in I$ but $y\notin I$. Then the rays at $y$ separate $I$.
	\end{proof}
	
	\begin{rmk}
		To see the necessity of the assumptions, consider $\mathbb Q$ and $\mathbb Z$ \resp which are both totally disconnected.
	\end{rmk}
	
	
	\begin{prp}[Intermediate value]
		Any continuous function from a connected space to a totally ordered space obeys intermediate value property.
	\end{prp}
	
	\begin{proof}
		Let $X$ be connected and $Y$ ordered, and $f\colon X\to Y$ be continuous. Suppose $f$ doesn't obey intermediate value property. Then take $x_1, x_2\in X$ and $y\in Y$ such that $y$ lies between $f(x_1)$ and $f(x_2)$ and yet $y\notin f(X)$. \Wlogg, let $f(x_1) < f(x_2)$ (since $y$ lies between them, they can't be equal; \uline{totality used}) so that $f(x_1) < y < f(x_2)$. Now, $f^{-1}((-\infty, y))$ and $f^{-1}((y, +\infty))$ form a nonempty (they contain $x_1$, $x_2$ \resp) open separation of $X$, violating the connectedness of $X$.
	\end{proof}
	
	
	
	

\section{Separation Axioms}

	\begin{lem}[$T_1$ spaces]\label{LEM: t1 spaces}
		In a space, singletons are closed $\iff$ any two distinct points can be separated by open sets that don't contain the other.
	\end{lem}
	
	\begin{proof}
		Let the space in question be $X$.
		
		``$\Rightarrow$'': Let $x$, $y$ be distinct. Then $X\setminus\{y\}$ and $X\setminus\{y\}$ separate $x$, $y$ as required.
		
		``$\Leftarrow$'': Let $x\in X$. We show that $X\setminus\{x\}$ is open, which follows easily.
	\end{proof}
	

	\begin{prp}\label{PRP: cont func into Hausdorff uniquely by its vals on dense subset}
		A continuous function taking values in a Hausdorff codomain is completely determined by its values on a dense subset of the domain.
	\end{prp}
	
	\begin{proof}
		Let $f, g\colon X\to Y$ be continuous with $Y$ Hausdorff, agreeing on a dense subset $D\subseteq X$ and yet not on $x\in X$. Since \uline{$Y$ Hausdorff}, separate $f(x)$ and $g(x)$ via opens $V$ and $W$. Then $f^{-1}(V)\cap g^{-1}(W)$ is an \onbd of $x$, and thus intersects the \uline{dense $D$}, say at $y$. But then $V\ni f(y) = g(y)\in W$, a contradiction.
	\end{proof}
	
	\begin{rmk}
		To see the necessity of the Hausdorff codomain (and that just $T_1$ is not enough), consider the function on $\mathbb R$ which swaps two distinct points. Then this is continuous\footnote{
			More generally, for any set $X$, any bijection $X\to X_\text{cofin}$ is continuous if singletons are closed in the domain.
		} with the codomain under cofinite topology.
	\end{rmk}

	



\section{Countability and Separability}

	\begin{lem}
		A second countable space is separable and first countable.
	\end{lem}
	
	\begin{proof}
		Choosing a point\myMargin{
			\CC used.
		} from each of the sets from a countable base yields a countable dense set.
	\end{proof}
	
	\begin{rmk}
		The converse is not true (however, see \myRef{LEM: separable metric spaces are second countable}): Consider the Sorgenfrey line, \ie, the lower limit topology on $\mathbb R$ generated by the basic open sets of the form $[a, b)$. Any base of this topology must contain for each $x\in\mathbb R$, some set with $x$ being its \lub, and thus be uncountable.\myMargin{
			\AC used.
		}
	\end{rmk}
	
	\begin{center}
		\begin{tabular}{c|c|c|c}
			second countable & first countable & separable & \\
			\hline
			\mycheck & & & separable metric spaces\\
			\mycross & \mycheck & \mycheck & Sorgenfrey line\\
			\mycross & \mycheck & \mycross & nonseparable metric spaces\footnotemark\\
			& \mycross & \mycheck & cofinite on uncountable\\
			& \mycross & \mycross & cocountable on uncountable
		\end{tabular}\footnotetext{For instance, discrete metric on any uncountable set.}
	\end{center}
	

	\begin{prp}
		Any base of a second countable space contains a countable base.
	\end{prp}
	
	\begin{proof}
		Let $\mathscr B$, $\mathscr B'$ be bases of $X$ with $\mathscr B$ being countable. It suffices\myMargin{\CC used}.
		to show that each $U\in\mathscr B$ is a countable union in $\mathscr B'$. Thus, consider a $U\in\mathscr B$.\myMargin{Add a diagram.}
		Define $\mathscr V:= \{V\in\mathscr B : V\subseteq W'\subseteq U \text{ for some } W'\in\mathscr B'\}$. Now, for each $V\in\mathscr V$, one can choose\myMargin{\CC used.}
		a $W'_V\in\mathscr B$ such that $V\subseteq W'_V\subseteq U$. Now, just note that $U$ is the union of $W'_V$'s which are countably many.
	\end{proof}
	
	
	\begin{prp}
		Separability, and first and second countabilities are preserved under countable products.
	\end{prp}
	
	\begin{proof}
		\begin{mylist}
			\item Separability:
			Let $A_i$ be countable and dense in $X_i$ for $i = 1, 2, \ldots$.\myMargin{
				\CC used.
			} \Wlogg, let each $X_i$ be nonempty so that we may choose
			\myMargin{
				\CC used.
			} for each $i$, an $x_i\in X_i$. Then the union of the following sets forms a countable dense set in $\prod_i X_i$:
			\begin{mylist}
				\item $A_1\times\{x_2\}\times\{x_3\}\times\cdots$
				\item $A_1\times A_2\times \{x_3\}\times\{x_4\}\times\cdots$
				\item $A_1\times A_2\times A_3\times \{x_4\}\times\{x_5\}\times\cdots$
				\item[\vdots] 
			\end{mylist}
			
			\item First countability:
			Let $X_1, X_2, \ldots$ be first countable and let $x\in \prod_i X_i$. For each $i$, choose\myMargin{
				\CC used.
			} a countable local base $(B^{(i)}_j)_j$ at $x_i$. \Wlogg, let $(B^{(i)}_j)_j$ be decreasing for each $i$. Then the following sets form a local base at $x$:
			\begin{mylist}
				\item $\pi_1^{-1}(B^{(1)}_1)$
				\item $\pi_1^{-1}(B^{(1)}_2)\cap \pi_2^{-1}(B^{(2)}_2)$
				\item $\pi_1^{-1}(B^{(1)}_3)\cap \pi_2^{-1}(B^{(2)}_3)\cap \pi_3^{-1}(B^{(3)}_3)$
				\item[\vdots]
			\end{mylist}
			
			\item Second countability: 
			For $i = 1, 2, \ldots$, choose countable local bases $\mathscr B_i$'s for second countable $X_i$.\myMargin{
				\CC used.
			} Then the union of the following collections forms a countable base for $\prod_i X_i$:\footnote{Notation abused for $\pi_i^{-1}$ and $\cap$.}
			\begin{mylist}
				\item $\pi_1^{-1}(\mathscr B_1)$
				\item $\pi_1^{-1}(\mathscr B_1)\cap\pi_2^{-1}(\mathscr B_2)$
				\item $\pi_1^{-1}(\mathscr B_1)\cap\pi_2^{-1}(\mathscr B_2)\cap\pi_3^{-1}(\mathscr B_3)$
				\item[$\vdots$]\qedhere
			\end{mylist}
		\end{mylist}
	\end{proof}
	
	\begin{rmk}
		It turns out that \href{https://math.stackexchange.com/a/526460/673223}{Hewitt-Marczewski-Pondiczery theorem} implies that $\mathfrak c$-fold product also preserves separability.
		
		Preservation of first (and hence second) countability under uncountable products is not true: Consider an uncountable product of discrete $\{0, 1\}$.
	\end{rmk}
	
	
	\begin{prp}\label{PRP: continuity and closure in first countable}
		\leavevmode
		\begin{mylist}
			\item For a first countable domain, sequential continuity $\implies$ continuity.
			
			\item For a first countable space, closure is precisely the set of limits of sequences.
		\end{mylist}
	\end{prp}
	
	\begin{proof}
		\begin{mylist}
			\item Let $f\colon X\to Y$ be sequentially continuous at $c\in X$ with $X$ being first countable. Suppose $f$ is not continuous at $c$. Thus, take an \onbd $V$ of $f(c)$ such that $f(U)$ spills outside $V$ for each \onbd $U$ of $c$. Let $B_n$'s form a local base at $c$ and choose for each $n$,\myMargin{In both, \CC's usage can be avoided if $X$ is separable.}
			an $x_n\in B_n$ such that $f(x_n)\notin V$. But then $f(x_n)\not\to f(c)$ despite $x_n\to c$.
			
			\item Let $c\in\clos A\setminus A$ and let $B_n$'s form a local base at $c$. Then for each $n$, choose $x_n\in B_n\cap A$. Then $(x_n)$ is a sequence in $A$ converging to $c$.\qedhere
		\end{mylist}
	\end{proof}
	
	\begin{rmk}
		\begin{mylist}
			\item Any function from a co-countable topology is sequentially continuous, and yet needn't be continuous, for instance, $\id_X\colon X_\text{co-count}\to X_\text{discr}$ for any uncountable $X$.
			
			\item For the cocountable topology on an uncountable set, the closure of any nonempty open set in the cocountable topology is the whole space.
		\end{mylist}
	\end{rmk}
	
	\begin{cor}\label{COR: first countable T1 topologies via their sequences}
		A first countable topology is determined by convergence.\footnote{
			That is if $\tau_1$, $\tau_2$ are first countable topologies on $X$ with $x_i\to c$ in $\tau_1$ $\iff$ $x_i\to c$ in $\tau_2$, then $\tau_1 = \tau_2$.
		} Further, if the space is $T_1$ as well, then specifying just the \cgt sequences suffices.
	\end{cor}
	
	\begin{proof}
		Just note that in a \uline{$T_1$ space}, $x_i\to c$ $\iff$ $x_1, c, x_2, c, x_3, c, \ldots$ is \cgt.
	\end{proof}
	
	\begin{rmk}
		\begin{mylist}
			\item To see the necessity of first countability, note that cocountable and discrete topologies have the same convergent sequences and their limits. (Note that discrete is first countable.)
			
			\item To see the necessity of $T_1$, consider the Sierpiński and indiscrete topologies on $\{0, 1\}$.
		\end{mylist}
	\end{rmk}