\chapter{Topology}

\begin{conv}
	Unless stated otherwise,
	\begin{assmplist}
		\item $X$, $Y$ will be topological spaces.
		
		\item Subsets of topological spaces will be considered under subspace topology.
		
		\item Product of topological spaces will be considered under product the topology.
	\end{assmplist}
\end{conv}


\section{Subspaces and Bases}

	\begin{lem}
		$\mathscr B$ is a base iff the arbitrary unions in $\mathscr B$ form a topology.\myMargin{``$\Rightarrow$'' requires \AC.}
	\end{lem}
	
	\begin{lem}\label{LEM: subspaces and bases}
		\leavevmode
		\begin{mylist}
			\item ``Being a subspace of'' is transitive.
			
			\item\label{LEMii: subspaces and bases} (Sub)base of a subspace can be obtained from that of the parent space.
		\end{mylist}
	\end{lem}



\section{Product Topology}

	From \ref{LEMii: subspaces and bases} of \myRef{LEM: subspaces and bases}, we immediately conclude:

	\begin{lem}
		Taking products and subspaces are compatible.
	\end{lem}
	
	\begin{rmk}
		This holds for box topology as well.
	\end{rmk}
	
	\begin{lem}
		Closure of a product is the product of closures.
	\end{lem}
	
	\begin{proof}
		Let $A_i\subseteq X_i$. We show $\clos{\prod_i A_i} = \prod_i\clos{A_i}$.
		
		``$\subseteq$'': Suffice to show that $\prod_i F_i$ is closed for $F_i$'s closed in $X_i$'s. Let $(x_i)\notin\prod_i F_i$, say $x_{i_0}\notin F_{i_0}$. Then take an \onbd $U_{i_0}$ of $x_{i_0}$ disjoint from $F_{i_0}$. Now, $\pi_{i_0}^{-1}(U_{i_0})$ is an \onbd of $(x_i)$ that is disjoint from $\prod_i F_i$.
		
		``$\supseteq$'': Let $U := \bigcap_{j\in J}\pi_j^{-1}(U_j)$ be an \onbd of $(x_i)\in\RHS$, where $J$ is finite and each $U_j$ is open. Then each $U_j$ is an \onbd of $x_j$ and hence intersects $A_j$. Thus $U$ intersects $\prod_i A_i$.\myMargin{No choice \reqd.}
	\end{proof}
	
	\begin{rmk}
		The same holds for box topology as well; however \AC will be required for ``$\supseteq$''.
	\end{rmk}
	
	


\section{Order Topology}
If $X$ is totally ordered, then the \defn{order topology} on it is generated by such sets:
\begin{rmklist}
	\item $(a, b)$;
	\item $[\min X, b)$ if $X$ has a minimum \elt; and,
	\item $(a, \max X]$ if $X$ has a maximum \elt.
\end{rmklist}

	\begin{lem}\label{LEM: triv things on order topo}
		\leavevmode
		\begin{mylist}
			\item Open rays are open in order topology.
			\item Order topology is Hausdorff.
			\item\label{LEMiii: triv things on order topo} Topology induced from inherited order is coarser than the subspace topology.
		\end{mylist}
	\end{lem}
	
	\begin{proof}
		\begin{mylist}
			\item Let's show for right-rays. In case there's a largest \elt, then it's clear. If not, then $(a, +\infty) = \bigcup_y (a, y)$, which is open.
			
			\item Let $x < y$. If there's a $z$ between them, then $(-\infty, z)$ and $(z, +\infty)$ separate them. Otherwise, $(-\infty, y)$ and $(x, +\infty)$ do.
			
			\item Obviously.\qedhere
		\end{mylist}
	\end{proof}
	
	\begin{rmk}
		To see strict inclusion in \ref{LEMiii: triv things on order topo}, consider $\{-1\}\cup\{1/n : n\ge 1\}\subseteq \mathbb R$.
	\end{rmk}
	
	



\section{Denseness}

	\begin{lem}\label{LEM: denseness is transitive}
		``Being dense'' is transitive.
	\end{lem}
	
	\begin{proof}
		Let $A\subseteq B\subseteq X$ with $A$ dense in $B$ and $B$ dense in $X$. Let $U$ be a nonempty open in $X$. Then \ul{$B$ being dense}, intersects $U$ so that $U\cap B$ is a nonempty open in $B$ and thus is intersected by the \ul{dense $A$} $\wimplies$ $A$ intersects $U$.
	\end{proof}
	
	\begin{lem}\label{LEM: dense on a subset}
		Let $A, B\subseteq X$. Then \tfh:
		\begin{mylist}
			\item $B\cap A$ is dense in $B$ $\implies$ $B\subseteq \clos A$.
			
			\item\label{LEMii: dense on a subset} The converse holds if $B$ is open.
		\end{mylist}
	\end{lem}
	
	\begin{proof}
		\begin{mylist}
			\item We have $B = \cl_B(B\cap A)\subseteq B\cap \clos{B\cap A}\subseteq\clos{B\cap A}\subseteq\clos A$.
			
			\item We need to show that $\cl_B(B\cap A) = B$. Indeed, if $F$ is any closed such that $B\cap A\subseteq B\cap F$, then $B\subseteq F$ (otherwise, take $x\in B\setminus F\wimplies x\in B\setminus A\wimplies x\in B\setminus \clos A$ for \ul{$B$ is open}, contradicting $B\subseteq \clos A$).\qedhere
		\end{mylist}
	\end{proof}
	
	\begin{rmk}
		To see the necessity of openness of $B$ in \ref{LEMii: dense on a subset}, consider $A = \{1, 1/2, \ldots\}$ and $B = \{0\}$.
	\end{rmk}
	
	
	
\subsection{Nowhere dense sets}
	
	\myRef{LEM: dense on a subset} gives insight as to why nowhere dense sets are called so---they are dense on no nonempty \emph{open} set. On the other hand, dense sets are dense on the whole space.
	
	\begin{lem}
		Let $U$ be open in $X$ and $A\subseteq X$. Then \tfae:
		\begin{mylist}
			\item $U\subseteq \clos A$.
			
			\item Every nonempty open subset contained in $U$ intersects $\clos A$.
			
			\item Every nonempty open subset contained in $U$ intersects $A$.
		\end{mylist}
	\end{lem}
	
	\begin{cor}\label{COR: characterizing nowhere dense}
		\Tfae for a subset $A$ of $X$:
		\begin{mylist}
			\item $X\setminus \overline A$ is dense.
			
			\item $A$ is nowhere dense.
			
			\item Each nonempty open set contains a nonemtpy open subset disjoint from $\overline A$.
			
			\item Each nonempty open set contains a nonemtpy open subset disjoint from $A$.
		\end{mylist}
	\end{cor}
	
	Subsets of a topological space that are countable unions of nowhere dense sets are called \defn{first category} or \defn{meagre} sets. Others are called \defn{second category} sets.
	
	\begin{rmk}
		In $\mathbb R$:\footnote{Nonmeagre-ness can be concluded by Baire's category theorem (\myRef{THM: BCT}).}
		\begin{center}
			\begin{tabular}{r|cc}
				& meagre & nonmeagre\\
				\hline
				dense & $\mathbb Q$ & $\mathbb R$\\
				nondense & $\emptyset$ & $[0, 1]$
			\end{tabular}
		\end{center}
	\end{rmk}
	
	\begin{lem}
		If $F_1, F_2, \ldots$ are closed in $X$ with $X\setminus\bigcup_i F_i$ dense, then each $F_i$ is nowhere dense.
	\end{lem}
	
	\begin{rmk}
		Baire's category theorem (\myRef{THM: BCT}) gives a converse to above, stating that complements of meagre sets are dense in a complete metric space.
	\end{rmk}
	
	\begin{prp}\label{PRP: versions of BCT}
		In a topological space, \tfae:
		\begin{mylist}
			\item\label{PRPi: versions of BCT} Complements of meagre sets are dense.
			\item\label{PRPii: versions of BCT} Countable intersections of open dense sets are dense.
		\end{mylist}
	\end{prp}
	
	\begin{proof}
		`` $\Rightarrow$ '': Let $U_1, U_2, \cdots$ be open dense. Now, $\bigcap_i U_i \which= X\setminus\bigcup_i(X\setminus U_i)$ is dense if each $X\setminus U_i$ is nowhere dense $\wimpliedby$ $X\setminus(\overline{X\setminus U_i})\which= U_i$ (since \ul{$U_i$ open}) is \ul{dense}, which is true.
		
		`` $\Leftarrow$ '': Let $A_1, A_2, \ldots$ be nowhere dense. Then each $X\setminus\overline A_i$ is dense $\wimplies$ $\bigcap_i(X\setminus \overline A_i)\which= X\setminus\bigcup_i\overline A_i$ is dense $\wimplies$ $X\setminus\bigcup_i A_i$ is dense as well, being a larger set.
	\end{proof}
	
	
	
	
\section{Connectedness}

	\begin{lem}[Characterizing disconnectedness]
		$E\subseteq X$ is disconnected $\iff$ $E$ can be written as a union of two nonempty subsets $A$, $B$ of $X$ such that $\overline A\cap B = \emptyset = A\cap\overline B$.
	\end{lem}
	
	\begin{proof}
		``$\Rightarrow$'': Take $U$, $V$ open in $X$ such that $E\cap U$, $E\cap V$ are nonempty, $E\subseteq U\cup V$, and $E\cap U\cap V = \emptyset$. Now put $A := E\cap U$ and $B := E\cap V$. Then $\overline A\cap B\subseteq \overline{E\cap U}\cap V = \emptyset$.
		
		``$\Leftarrow$'': Take $U := X\setminus \overline A$ and $V := X\setminus \overline B$. Then $B\subseteq U$ and $A\subseteq V$ so that both are nonempty and $E\subseteq U\cup V$. Also, $E\cap U\cap V = E\setminus(\overline A\cup\overline B) = \emptyset$.
	\end{proof}
	
	\begin{prp}[Linear continua are connected]\label{PRP: linear continua re connected}
		Let $X$ be a totally ordered set such that \tfh:
		\begin{assmplist}
			\item Any nonempty subset that is bounded above has a \lub
			\item Any two points have a point in between them.
		\end{assmplist}
		Then under the order topology on $X$, connected subsets of $X$ are precisely its convex subsets.\footnote{Recall that a convex subset of an ordered set is any set $I$ such that $[x, y]\subseteq I$ whenever $x, y\in I$ with $x\le y$.}
	\end{prp}
	
	\begin{proof}
		Suppose $I\subseteq X$ is convex, and yet separated by opens $U$, $V$. Take $a\in U\cap I$ and $b\in V\cap I$. \Wlogg, assume $a < b$ (\ul{the order is total}) so that $[a, b]\subseteq I$ (since \ul{$I$ is convex}). Note that $U$, $V$ also form a separation of $[a, b]$. Since $U\cap [a, b]$ is nonempty and bounded, let $c$ be its \ul{\lub} Clearly, $c\in[a, b]$ so that there are two cases:
		\begin{mylist}
			\item[$c\in U$:] Take a basic interval $J\subseteq U$ containing $c$. Note that $c < b$ (since $b\in V$) so that $J\supseteq [c, d)$ for some $d > c$. Hence, $U\cap [a, b]\which\supseteq J\cap [a, b]\supseteq[c, d)\cap[c, b] \which= [c, d_1)$ where $d_1 :=  \min(d, b) > c$. Now take \ul{$e$ between $c$ and $d_1$}. Then $e\in U\cap[a, b]$ despite $e > c$.\myMargin{Add a diagram!}
			
			\item[$c\in V$:] Again take a basic interval $J\subseteq V$ containing $c$. This time, $c > a$ (as $a\in U$) so that $J\supseteq (d, c]$ for some $d < c$. Thus, $V\cap[a, b]\which\supseteq J\cap[a, b]\supseteq (d, c]\cap [a, c]\which=(d_1, c]$ where $d_1 := \max(d, a) < c$. Now, take \ul{$e$ between $d_1$ and $c$}. Then $e$ is an \ub for $U\cap[a, b]$ greater than $c$:
			\begin{subproof}
				If $x\in U\cap[a, b]$ is greater than $e$, then $x\in (e, c]\subseteq(d_1, c]\subseteq V$.
			\end{subproof}
		\end{mylist}
		
		Conversely, if $I$ is not convex, then take $x < y < z$ such that $z, z\in I$ but $y\notin I$. Then the rays at $y$ separate $I$.
	\end{proof}
	
	\begin{rmk}
		To see the necessity of the assumptions, consider $\mathbb Q$ and $\mathbb Z$ \resp which are both totally disconnected.
	\end{rmk}
	
	
	
	

\section{Separation Axioms}

	\begin{prp}\label{PRP: cont func into Hausdorff uniquely by its vals on dense subset}
		A continuous function on a Hausdorff domain is completely determined by its values on a dense subset of the domain.
	\end{prp}
	
	\begin{proof}
		Let $f, g\colon X\to Y$ be continuous with $Y$ Hausdorff, agreeing on a dense subset $D\subseteq X$ and yet not on $x\in X$. Since \ul{$Y$ Hausdorff}, separate $f(x)$ and $g(x)$ via opens $V$ and $W$. Then $f^{-1}(V)\cap g^{-1}(W)$ is an \onbd of $x$, and thus intersects the \ul{dense $D$}, say at $y$. But then $V\ni f(y) = g(y)\in W$, a contradiction.
	\end{proof}
	
	\begin{rmk}
		To see the necessity of the Hausdorff codomain (and that just $T_1$ is not enough), consider the identity function on $\mathbb R$ except that it swaps two distinct points. Then this is continuous\footnote{
			More generally, for any set $X$, any bijection $X\to X_\text{cofin}$ is continuous if singletons are closed in the domain.
		} with the codomain under cofinite topology.
	\end{rmk}

	



\section{Countability and Separability}

	\begin{lem}
		A second countable space is separable and first countable.
	\end{lem}
	
	\begin{proof}
		Choosing a point\myMargin{
			\CC used.
		} from each of the sets from a countable base yields a countable dense set.
	\end{proof}
	
	\begin{rmk}
		The converse is not true (however, see \myRef{LEM: separable metric spaces are second countable}): Consider the Sorgenfrey line, \ie, the lower limit topology on $\mathbb R$ generated by the basic open sets of the form $[a, b)$. Any base of this topology must contain for each $x\in\mathbb R$, some set with $x$ being its \lub, and thus be uncountable.\myMargin{
			\AC used.
		}
	\end{rmk}
	
	\begin{center}
		\begin{tabular}{c|c|c|c}
			second countable & first countable & separable & \\
			\hline
			\mycheck & & & separable metric spaces\\
			\mycross & \mycheck & \mycheck & Sorgenfrey line\\
			\mycross & \mycheck & \mycross & nonseparable metric spaces\footnotemark\\
			& \mycross & \mycheck & cofinite on uncountable\\
			& \mycross & \mycross & cocountable on uncountable
		\end{tabular}\footnotetext{For instance, discrete topology on any uncountable set.}
	\end{center}
	

	\begin{prp}
		Any base of a second countable space contains a countable base.
	\end{prp}
	
	\begin{proof}
		Let $\mathscr B$, $\mathscr B'$ be bases of $X$ with $\mathscr B$ being countable. It suffices\myMargin{\CC used}.
		to show that each $U\in\mathscr B$ is a countable union in $\mathscr B'$. Thus, consider a $U\in\mathscr B$.\myMargin{Add a diagram.}
		Define $\mathscr V:= \{V\in\mathscr B : V\subseteq W'\subseteq U \text{ for some } W'\in\mathscr B'\}$. Now, for each $V\in\mathscr V$, one can choose\myMargin{\CC used.}
		a $W'_V\in\mathscr B$ such that $V\subseteq W'_V\subseteq U$. Now, just note that $U$ is the union of $W'_V$'s which are countably many.
	\end{proof}
	
	
	\begin{prp}
		Second countability is preserved under countable products.
	\end{prp}
	
	\begin{proof}
		For $i = 1, 2, \ldots$, let $\mathscr B_i$ be a countable base for $X_i$. Then the collection of the following sets forms a base for $\prod_i X_i$:\footnote{Notation abused for $\pi_i^{-1}$ and $\cap$.}
		\begin{mylist}
			\item $\pi_1^{-1}(\mathscr B_1)$
			\item $\pi_1^{-1}(\mathscr B_1)\cap\pi_2^{-1}(\mathscr B_2)$
			\item $\pi_1^{-1}(\mathscr B_1)\cap\pi_2^{-1}(\mathscr B_2)\cap\pi_3^{-1}(\mathscr B_3)$
			\item[$\vdots$]\qedhere
		\end{mylist}
	\end{proof}
	
	\begin{rmk}
		Preservation not guaranteed under uncountable products: Consider an uncountable product of discrete $\{0, 1\}$.
	\end{rmk}
	
	
	\begin{prp}\label{PRP: continuity and closure in first countable}
		\leavevmode
		\begin{mylist}
			\item For a first countable domain, sequential continuity $\implies$ continuity.
			
			\item For a first countable space, closure is precisely the set of limits of sequences.
		\end{mylist}
	\end{prp}
	
	\begin{proof}
		\begin{mylist}
			\item Let $f\colon X\to Y$ be sequentially continuous at $c\in X$ with $X$ being first countable. Suppose $f$ is not continuous at $c$. Thus, take an \onbd $V$ of $f(c)$ such that $f(U)$ spills outside $V$ for each \onbd $U$ of $c$. Let $B_n$'s form a local base at $c$ and choose for each $n$,\myMargin{In both, \CC's usage can be avoided if $X$ is separable.}
			an $x_n\in B_n$ such that $f(x_n)\notin V$. But then $f(x_n)\not\to f(c)$ despite $x_n\to c$.
			
			\item Let $c\in\clos A\setminus A$ and let $B_n$'s form a local base at $c$. Then for each $n$, choose $x_n\in B_n\cap A$. Then $(x_n)$ is a sequence in $A$ converging to $c$.\qedhere
		\end{mylist}
	\end{proof}
	
	\begin{rmk}
		\begin{mylist}
			\item Any function from a co-countable topology is sequentially continuous, and yet needn't be continuous, for instance, $\id_X\colon X_\text{co-count}\to X_\text{discr}$ for any uncountable $X$.
			
			\item For the cocountable topology on an uncountable set, the closure of any nonempty open set in the cocountable topology is the whole space.
		\end{mylist}
	\end{rmk}
	
	\begin{cor}\label{COR: first countable T1 topologgies via their sequences}
		A first countable topology is determined by convergence.\footnote{That is $x_i\to c$ in $\tau_1$ iff $x_i\to c$ in $\tau_2$.} Further, if the space is $T_1$ as well, then specifying just the \cgt sequences suffices.
	\end{cor}
	
	\begin{proof}
		Just note that in a \ul{$T_1$ space}, $x_i\to c$ iff $x_1, c, x_2, c, x_3, c, \ldots$ is \cgt.
	\end{proof}
	
	\begin{rmk}
		\begin{mylist}
			\item To see the necessity of first countability, note that cocountable and discrete topologies have the same convergent sequences and their limits. (Note that discrete is first countable.)
			
			\item To see the necessity of $T_1$, consider the Sierpiński and indiscrete topologies on $\{0, 1\}$.
		\end{mylist}
	\end{rmk}