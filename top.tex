\chapter{Topology}

\begin{conv}
	Unless stated otherwise,
	\begin{assmplist}
		\item $X$, $Y$ will be topological spaces.
		
		\item Subsets of topological spaces will be considered under subspace topology.
		
		\item Product of topological spaces will be considered under product the topology.
	\end{assmplist}
\end{conv}


\section{Subspaces and Bases}

	\begin{lem}
		$\mathscr B$ is a base iff the arbitrary unions in $\mathscr B$ form a topology.\myMargin{``$\Rightarrow$'' requires \AC.}
	\end{lem}
	
	\begin{lem}\label{LEM: subspaces and bases}
		\leavevmode
		\begin{mylist}
			\item ``Being a subspace of'' is transitive.
			
			\item\label{LEMii: subspaces and bases} (Sub)base of a subspace can be obtained from that of the parent space.
		\end{mylist}
	\end{lem}



\section{Product Topology}

	From \ref{LEMii: subspaces and bases} of \myRef{LEM: subspaces and bases}, we immediately conclude:

	\begin{lem}
		Taking products and subspaces are compatible.
	\end{lem}
	
	\begin{rmk}
		This holds for box topology as well.
	\end{rmk}
	
	\begin{lem}
		Closure of a product is the product of closures.
	\end{lem}
	
	\begin{proof}
		Let $A_i\subseteq X_i$. We show $\clos{\prod_i A_i} = \prod_i\clos{A_i}$.
		
		``$\subseteq$'': Suffice to show that $\prod_i F_i$ is closed for $F_i$'s closed in $X_i$'s. Let $(x_i)\notin\prod_i F_i$, say $x_{i_0}\notin F_{i_0}$. Then take an \onbd $U_{i_0}$ of $x_{i_0}$ disjoint from $F_{i_0}$. Now, $\pi_{i_0}^{-1}(U_{i_0})$ is an \onbd of $(x_i)$ that is disjoint from $\prod_i F_i$.
		
		``$\supseteq$'': Let $U := \bigcap_{j\in J}\pi_j^{-1}(U_j)$ be an \onbd of $(x_i)\in\RHS$, where $J$ is finite and each $U_j$ is open. Then each $U_j$ is an \onbd of $x_j$ and hence intersects $A_j$. Thus $U$ intersects $\prod_i A_i$.\myMargin{No choice \reqd.}
	\end{proof}
	
	\begin{rmk}
		The same holds for box topology as well; however \AC will be required for ``$\supseteq$''.
	\end{rmk}
	
	


\section{Order Topology}
If $X$ is totally ordered, then the \defn{order topology} on it is generated by such sets:
\begin{rmklist}
	\item $(a, b)$;
	\item $[\min X, b)$ if $X$ has a minimum \elt; and,
	\item $(a, \max X]$ if $X$ has a maximum \elt.
\end{rmklist}

	\begin{lem}\label{LEM: triv things on order topo}
		\leavevmode
		\begin{mylist}
			\item Open rays are open in order topology.
			\item Order topology is Hausdorff.
			\item\label{LEMiii: triv things on order topo} Topology induced from inherited order is coarser than the subspace topology.
		\end{mylist}
	\end{lem}
	
	\begin{proof}
		\begin{mylist}
			\item Let's show for right-rays. In case there's a largest \elt, then it's clear. If not, then $(a, +\infty) = \bigcup_y (a, y)$, which is open.
			
			\item Let $x < y$. If there's a $z$ between them, then $(-\infty, z)$ and $(z, +\infty)$ separate them. Otherwise, $(-\infty, y)$ and $(x, +\infty)$ do.
			
			\item Obviously.\qedhere
		\end{mylist}
	\end{proof}
	
	\begin{rmk}
		To see strict inclusion in \ref{LEMiii: triv things on order topo}, consider $\{-1\}\cup\{1/n : n\ge 1\}\subseteq \mathbb R$.
	\end{rmk}




\section{Nowhere Dense Sets}

	First, an easy characterization:
	
	\begin{lem}\label{LEM: characterizing nowhere dense}
		\Tfae for a subset $A$ of $X$:
		\begin{mylist}
			\item\label{CORi: characterizing nowhere dense} $X\setminus \overline A$ is dense.
			
			\item $A$ is nowhere dense.
			
			\item Each nonempty open set contains a nonemtpy open subset disjoint from $\overline A$.
			
			\item\label{CORiv: characterizing nowhere dense} Each nonempty open set contains a nonemtpy open subset disjoint from $A$.
		\end{mylist}
	\end{lem}
	
	\begin{proof}
		We only show ``\ref{CORiv: characterizing nowhere dense} $\Rightarrow$ \ref{CORi: characterizing nowhere dense}'', the rest being trivial: Suppose $X\setminus \overline A$ is not dense, thus getting a nonempty open $U\subseteq \overline A$. But then each nonempty open set contained in $U$ intersects $A$, a contradiction.
	\end{proof}
	
	Subsets of a topological space that are countable unions of nowhere dense sets are called \defn{first category} or \defn{meagre} sets. Others are called \defn{second category} sets.
	
	\begin{rmk}
		In $\mathbb R$:\footnote{Nonmeagre-ness can be concluded by Baire's category theorem (\myRef{THM: BCT}).}
		\begin{center}
			\begin{tabular}{r|cc}
				& meagre & nonmeagre\\
				\hline
				dense & $\mathbb Q$ & $\mathbb R$\\
				nondense & $\emptyset$ & $[0, 1]$
			\end{tabular}
		\end{center}
	\end{rmk}
	
	\begin{lem}
		If $F_1, F_2, \ldots$ are closed in $X$ with $X\setminus\bigcup_i F_i$ dense, then each $F_i$ is nowhere dense.
	\end{lem}
	
	\begin{rmk}
		Baire's category theorem (\myRef{THM: BCT}) gives a converse to above, stating that complements of meagre sets are dense in a complete metric space.
	\end{rmk}
	
	\begin{prp}\label{PRP: versions of BCT}
		In a topological space, \tfae:
		\begin{mylist}
			\item\label{PRPi: versions of BCT} Complements of meagre sets are dense.
			\item\label{PRPii: versions of BCT} Countable intersections of open dense sets are dense.
		\end{mylist}
	\end{prp}
	
	\begin{proof}
		`` $\Rightarrow$ '': Let $U_1, U_2, \cdots$ be open dense. Now, $\bigcap_i U_i \which= X\setminus\bigcup_i(X\setminus U_i)$ is dense if each $X\setminus U_i$ is nowhere dense $\wimpliedby$ $X\setminus(\overline{X\setminus U_i})\which= U_i$ (since \ul{$U_i$ open}) is \ul{dense}, which is true.
		
		`` $\Leftarrow$ '': Let $A_1, A_2, \ldots$ be nowhere dense. Then each $X\setminus\overline A_i$ is dense $\wimplies$ $\bigcap_i(X\setminus \overline A_i)\which= X\setminus\bigcup_i\overline A_i$ is dense $\wimplies$ $X\setminus\bigcup_i A_i$ is dense as well, being a larger set.
	\end{proof}
	
	
	
	
\section{Connectedness}

	\begin{lem}[Characterizing disconnectedness]
		$E\subseteq X$ is disconnected $\iff$ $E$ can be written as a union of two nonempty subsets $A$, $B$ of $X$ such that $\overline A\cap B = \emptyset = A\cap\overline B$.
	\end{lem}
	
	\begin{proof}
		``$\Rightarrow$'': Take $U$, $V$ open in $X$ such that $E\cap U$, $E\cap V$ are nonempty, $E\subseteq U\cup V$, and $E\cap U\cap V = \emptyset$. Now put $A := E\cap U$ and $B := E\cap V$. Then $\overline A\cap B\subseteq \overline{E\cap U}\cap V = \emptyset$.
		
		``$\Leftarrow$'': Take $U := X\setminus \overline A$ and $V := X\setminus \overline B$. Then $B\subseteq U$ and $A\subseteq V$ so that both are nonempty and $E\subseteq U\cup V$. Also, $E\cap U\cap V = E\setminus(\overline A\cup\overline B) = \emptyset$.
	\end{proof}
	
	\begin{prp}[Linear continua are connected]\label{PRP: linear continua re connected}
		Let $X$ be a totally ordered set such that \tfh:
		\begin{assmplist}
			\item Any nonempty subset that is bounded above has a \lub
			\item Any two points have a point in between them.
		\end{assmplist}
		Then under the order topology on $X$, connected subsets of $X$ are precisely its convex subsets.\footnote{Recall that a convex subset of an ordered set is any set $I$ such that $[x, y]\subseteq I$ whenever $x, y\in I$ with $x\le y$.}
	\end{prp}
	
	\begin{proof}
		Suppose $I\subseteq X$ is convex, and yet separated by opens $U$, $V$. Take $a\in U\cap I$ and $b\in V\cap I$. \Wlog, assume $a < b$ (\ul{the order is total}) so that $[a, b]\subseteq I$ (since \ul{$I$ is convex}). Note that $U$, $V$ also form a separation of $[a, b]$. Since $U\cap [a, b]$ is nonempty and bounded, let $c$ be its \ul{\lub} Clearly, $c\in[a, b]$ so that there are two cases:
		\begin{mylist}
			\item[$c\in U$:] Take a basic interval $J\subseteq U$ containing $c$. Note that $c < b$ (since $b\in V$) so that $J\supseteq [c, d)$ for some $d > c$. Hence, $U\cap [a, b]\which\supseteq J\cap [a, b]\supseteq[c, d)\cap[c, b] \which= [c, d_1)$ where $d_1 :=  \min(d, b) > c$. Now take \ul{$e$ between $c$ and $d_1$}. Then $e\in U\cap[a, b]$ despite $e > c$.\myMargin{Add a diagram!}
			
			\item[$c\in V$:] Again take a basic interval $J\subseteq V$ containing $c$. This time, $c > a$ (as $a\in U$) so that $J\supseteq (d, c]$ for some $d < c$. Thus, $V\cap[a, b]\which\supseteq J\cap[a, b]\supseteq (d, c]\cap [a, c]\which=(d_1, c]$ where $d_1 := \max(d, a) < c$. Now, take \ul{$e$ between $d_1$ and $c$}. Then $e$ is an \ub for $U\cap[a, b]$ greater than $c$:
			\begin{subproof}
				If $x\in U\cap[a, b]$ is greater than $e$, then $x\in (e, c]\subseteq(d_1, c]\subseteq V$.
			\end{subproof}
		\end{mylist}
		
		Conversely, if $I$ is not convex, then take $x < y < z$ such that $z, z\in I$ but $y\notin I$. Then the rays at $y$ separate $I$.
	\end{proof}
	
	\begin{rmk}
		To see the necessity of the assumptions, consider $\mathbb Q$ and $\mathbb Z$ \resp which are both totally disconnected.
	\end{rmk}
	



\section{Countability and Separability}
% First countable => sequential

	\begin{prp}
		Any base of a second countable space contains a countable base.
	\end{prp}
	
	\begin{proof}
		Let $\mathscr B$, $\mathscr B'$ be bases of $X$ with $\mathscr B$ being countable. It suffices\myMargin{\CC used}.
		to show that each $U\in\mathscr B$ is a countable union in $\mathscr B'$. Thus, consider a $U\in\mathscr B$.\myMargin{Add a diagram.}
		Define $\mathscr V:= \{V\in\mathscr B : V\subseteq W'\subseteq U \text{ for some } W'\in\mathscr B'\}$. Now, for each $V\in\mathscr V$, one can choose\myMargin{\CC used.}
		a $W'_V\in\mathscr B$ such that $V\subseteq W'_V\subseteq U$. Now, just note that $U$ is the union of $W'_V$'s which are countably many.
	\end{proof}
	
	
	\begin{prp}
		Second countability is preserved under countable products.
	\end{prp}
	
	\begin{proof}
		For $i = 1, 2, \ldots$, let $\mathscr B_i$ be a countable base for $X_i$. Then the collection of the following sets forms a base for $\prod_i X_i$:\footnote{Notation abused for $\pi_i^{-1}$ and $\cap$.}
		\begin{mylist}
			\item $\pi_1^{-1}(\mathscr B_1)$
			\item $\pi_1^{-1}(\mathscr B_1)\cap\pi_2^{-1}(\mathscr B_2)$
			\item $\pi_1^{-1}(\mathscr B_1)\cap\pi_2^{-1}(\mathscr B_2)\cap\pi_3^{-1}(\mathscr B_3)$
			\item[$\vdots$]\qedhere
		\end{mylist}
	\end{proof}
	
	\begin{rmk}
		Preservation not guaranteed under uncountable products: Consider an uncountable product of discrete $\{0, 1\}$.
	\end{rmk}
	
	
	\begin{prp}
		For a first countable domain, sequential continuity $\implies$ continuity.
	\end{prp}
	
	\begin{proof}
		Let $f\colon X\to Y$ be sequentially continuous at $c\in X$ with $X$ being first countable. Suppose $f$ is not continuous at $c$. Thus, take an \onbd $V$ of $f(c)$ such that $f(U)$ spills outside $V$ for each \onbd $U$ of $c$. Let $B_n$'s form an open base at $c$ and choose for each $n$,\myMargin{\CC used. However, if $X$ is separable, then no choice needed!}
		an $x_n\in B_n$ such that $f(x_n)\notin V$. But then $f(x_n)\not\to f(c)$ despite $x_n\to c$.
	\end{proof}
	
	\begin{rmk}
		For the necessity of first countability, note that any function from a co-countable topology is sequentially continuous, and yet needn't be continuous, for instance, $\id_X\colon X_\text{co-count}\to X_\text{discr}$ for any uncountable $X$.
	\end{rmk}